% ============================================================
% PodTeksT Audit — Preface
% ============================================================

\chapter*{Wstęp}
\addcontentsline{toc}{chapter}{Wstęp}

\begin{center}
\Large\itshape\color{PodBlue}
,,Najpierw dane, potem opinia --- nigdy odwrotnie.''
\end{center}

\vspace{12pt}

\section*{Cel audytu}

Niniejszy dokument stanowi kompleksowy, wieloagentowy audyt ekspercki aplikacji \podtekst --- analizatora konwersacji z~komunikatorów. Audyt obejmuje \textbf{wszystkie warstwy systemu}: od algorytmów ilościowych, przez pipeline analizy AI i~walidację psychologiczną, po parsery platform, interfejs użytkownika i~infrastrukturę.

Główne pytania badawcze:
\begin{enumerate}
  \item Czy algorytmy ilościowe (60+ metryk) są \textbf{poprawne matematycznie} i~wolne od bugów?
  \item Czy pipeline AI (4~pasy + moduły rozrywkowe) jest \textbf{psychologicznie uzasadniony}?
  \item Czy parsery (Messenger, WhatsApp, Instagram, Telegram, Discord) \textbf{poprawnie normalizują} dane wejściowe?
  \item Czy interfejs użytkownika jest \textbf{funkcjonalny, dostępny i~responsywny}?
  \item Gdzie są \textbf{krytyczne luki} w~bezpieczeństwie, walidacji i~jakości kodu?
  \item Co \textbf{warto dodać} w~przyszłych fazach rozwoju?
\end{enumerate}

\section*{Metodologia}

Audyt został przeprowadzony z~użyciem \textbf{3~niezależnych agentów analitycznych}, z~których każdy otrzymał specjalizowane zadanie eksploracji kodu źródłowego:

\begin{description}
  \item[\agentone] Dogłębna analiza silnika analizy ilościowej: 60+ metryk, wyniki wiralne (Compatibility, Interest, Ghost Risk, Delusion), indeks wzajemności, system odznak, wykrywanie burstów, metryki sieci, frazy charakterystyczne.
  \item[\agenttwo] Pełny audyt pipeline AI: 4~pasy analizy \gemini, próbkowanie wiadomości, kalibracja kontekstu relacji, frameworki psychologiczne (Big Five, attachment, MBTI), moduły rozrywkowe (Roast, Court Trial, Dating Profile, Simulator, Subtext Decoder, Delusion Quiz), bezpieczeństwo promptów.
  \item[\agentthree] Analiza parserów 5~platform, architektura UI, eksport PDF, rate limiting, walidacja danych wejściowych, kodowanie URL do udostępniania, dostępność.
\end{description}

Każdy agent przeczytał \textbf{kompletne pliki źródłowe} (nie streszczenia) i~wygenerował szczegółowy raport z~konkretnymi odniesieniami do linii kodu, formuł matematycznych i~schematów JSON.

\begin{warningbox}[title=Ważne zastrzeżenie]
Niniejszy audyt jest analizą \emph{statyczną} kodu źródłowego --- nie obejmuje testów dynamicznych (pentest, load testing, fuzzing). Oceny psychologiczne odnoszą się do \emph{założeń algorytmów}, nie do wyników na rzeczywistych danych. Audyt \textbf{nie} jest przeglądem klinicznym ani certyfikacją naukową.
\end{warningbox}

\section*{Konwencje dokumentu}

W~niniejszym raporcie stosowane są następujące konwencje wizualne:

\begin{criticalbox}[title=Błąd krytyczny]
Czerwone ramki oznaczają \textbf{błędy krytyczne} --- bugi wpływające na poprawność wyników, luki bezpieczeństwa lub problemy mogące prowadzić do awarii.
\end{criticalbox}

\begin{moderatebox}[title=Problem istotny]
Pomarańczowe ramki oznaczają \textbf{problemy istotne} --- kwestie kalibracji, arbitralne parametry lub brakujące walidacje, które nie powodują awarii, ale obniżają jakość wyników.
\end{moderatebox}

\begin{strengthbox}[title=Mocna strona]
Zielone ramki oznaczają \textbf{mocne strony} --- rozwiązania architektoniczne, algorytmy lub wzorce, które są szczególnie dobrze zaprojektowane.
\end{strengthbox}

\begin{infobox}[title=Informacja kontekstowa]
Niebieskie ramki zawierają \textbf{kontekst techniczny} --- odniesienia do plików, wyjaśnienia terminów lub dodatkowe informacje.
\end{infobox}

Identyfikatory błędów mają format \bugid{QUANT-01}, \bugid{AI-01}, \bugid{INFRA-01}. Identyfikatory problemów: \issueid{ENG-01}, \issueid{UI-01}. Pliki źródłowe: \filepath{src/lib/analysis/quantitative.ts}. Typy: \tstype{QuantitativeAnalysis}. Funkcje: \tsfunc{computeQuantitativeAnalysis()}.

\section*{Zakres}

\begin{table}[H]
\centering
\caption{Zakres audytu --- pliki i~moduły}
\begin{tabularx}{\textwidth}{l C{2cm} X}
\toprule
\textbf{Obszar} & \textbf{Plików} & \textbf{Kluczowe moduły} \\
\midrule
Silnik ilościowy & 11 & quantitative.ts, viral-scores.ts, badges.ts, network.ts, reciprocity.ts \\
Pipeline AI & 17 & gemini.ts, prompts.ts, qualitative.ts, subtext.ts, court-prompts.ts, delusion-quiz.ts \\
API Routes & 10 & analyze/, standup/, cps/, subtext/, court/, dating-profile/, simulate/ \\
Parsery & 7 & messenger.ts, whatsapp.ts, instagram.ts, telegram.ts, discord.ts, detect.ts \\
Interfejs & 15+ & analysis/[id]/page.tsx, SectionNavigator, StatsGrid, ShareCardGallery \\
Infrastruktura & 5 & rate-limit.ts, schemas.ts, encode.ts, pdf-export.ts, events.ts \\
\bottomrule
\end{tabularx}
\end{table}
