% ============================================================
% Rozdział 2 — Audyt Silnika Ilościowego
% Agent 1: Analiza algorytmów kwantytatywnych
% ============================================================

\chapter{Audyt Silnika Ilościowego}
\label{ch:audit-quant}

\begin{quote}
\textit{„Bez danych jesteś tylko kolejną osobą z~opinią."}\\
\hfill --- W. Edwards Deming
\end{quote}

\vspace{8pt}

Silnik ilościowy \podtekst stanowi fundament całej platformy analitycznej.
To właśnie on~--- działając wyłącznie po stronie klienta, bez jakiegokolwiek udziału AI~---
przetwarza surowe wiadomości na ponad 60~metryk statystycznych, od~prostych
liczników po wielowymiarowe wskaźniki wzajemności i~wiralności.
\index{silnik ilościowy}

Niniejszy rozdział prezentuje kompletne wyniki audytu przeprowadzonego przez \agentone.
Analiza obejmuje architekturę obliczeniową, poprawność matematyczną wzorów,
trafność psychologiczną metryk wiralnych, system odznak, wykrywanie burstów,
metryki sieciowe oraz moduł Communication Pattern Screening (CPS).

Zidentyfikowano \critical{5~błędów krytycznych}, \major{8~problemów istotnych}
oraz \minor{5~problemów drobnych}. Ocena końcowa silnika: \scoremed{6{,}5/10}.


% ============================================================
\section{Przegląd silnika}
\label{sec:quant-overview}
\index{architektura!silnik ilościowy}

% ------------------------------------------------------------
\subsection{Architektura single-pass O(n)}
\label{subsec:quant-arch}

Główna funkcja \tsfunc{computeQuantitativeAnalysis()} w~pliku
\filepath{src/lib/analysis/quantitative.ts} realizuje przetwarzanie
w~trzech fazach:

\begin{enumerate}
  \item \textbf{Inicjalizacja} --- tworzenie akumulatorów \tstype{PersonAccumulator}
    dla każdego uczestnika, zerowanie struktur heatmapy, map miesięcznych
    i~liczników sesji.
  \item \textbf{Główna pętla O(n)} --- pojedyncze przejście po tablicy wiadomości.
    Każda wiadomość aktualizuje: liczniki słów/znaków, emoji, reakcje, pytania,
    linki, media, czasy odpowiedzi, sesje, double texting, heatmapę, wolumeny
    miesięczne i~dzienne.
  \item \textbf{Post-processing} --- obliczanie metryk końcowych z~akumulatorów:
    mediany czasów odpowiedzi, trendy, bursts, viral scores, odznaki,
    catchphrases, metryki sieci i~indeks wzajemności.
\end{enumerate}

\begin{infobox}[title={Cel wydajnościowy}]
Deklarowany cel: \textbf{<200ms dla 50\,000 wiadomości}.
Architektura O(n) z~pojedynczym przejściem po tablicy wiadomości
jest właściwym wyborem. Operacje post-processingu (sortowanie, mediany,
regresja liniowa) mają złożoność O(m\,\text{log}\,m) gdzie $m \ll n$.
\end{infobox}

% ------------------------------------------------------------
\subsection{Platform-aware session gaps}
\label{subsec:quant-sessions}

Silnik rozróżnia platformy przy definiowaniu przerwy sesyjnej:

\begin{lstlisting}[style=podcode, caption={Przerwa sesyjna zależna od~platformy}]
function getSessionGapMs(platform: ParsedConversation['platform']): number {
  return platform === 'discord' ? 2 * 60 * 60 * 1000 : 6 * 60 * 60 * 1000;
}
\end{lstlisting}

Discord otrzymuje 2-godzinną przerwę sesyjną (konwersacje na serwerach są
bardziej rozproszone), pozostałe platformy~--- 6~godzin. To rozsądne
rozróżnienie, choć warto rozważyć parametryzację na poziomie konfiguracji
zamiast hardcoded wartości.

% ------------------------------------------------------------
\subsection{Moduły pomocnicze}
\label{subsec:quant-submodules}

Po refaktoryzacji TIER~3.3 silnik składa się z~następujących submodułów:

\begin{tabularx}{\textwidth}{l X}
  \toprule
  \textbf{Moduł} & \textbf{Odpowiedzialność} \\
  \midrule
  \filepath{quant/helpers.ts} & Ekstrakcja emoji, tokenizacja, statystyki, daty \\
  \filepath{quant/types.ts} & Typ \tstype{PersonAccumulator}, fabryka \\
  \filepath{quant/bursts.ts} & Detekcja burstów (7-dniowa średnia krocząca) \\
  \filepath{quant/trends.ts} & Trendy miesięczne (RT, długość wiad., inicjacje) \\
  \filepath{quant/reciprocity.ts} & Indeks wzajemności (4~wymiary) \\
  \filepath{viral-scores.ts} & Kompatybilność, zainteresowanie, ghost risk, delusion \\
  \filepath{badges.ts} & 15~odznak behawioralnych \\
  \filepath{catchphrases.ts} & Catchphrases (n-gramy) + Best Time to Text \\
  \filepath{network.ts} & Metryki sieci (graf interakcji, centralność) \\
  \filepath{constants.ts} & Stopwords, regresja liniowa \\
  \bottomrule
\end{tabularx}

\vspace{4pt}
Podział jest logiczny i~czytelny. Jedyny zarzut: \filepath{viral-scores.ts}
powinna znajdować się w~katalogu \filepath{quant/}, analogicznie do~pozostałych
submodułów.


% ============================================================
\section{Błędy krytyczne}
\label{sec:quant-critical}
\index{błędy krytyczne!silnik ilościowy}

Zidentyfikowano 5~błędów krytycznych wpływających na~poprawność wyników
prezentowanych użytkownikowi.

% ------------------------------------------------------------
\subsection{\bugid{QUANT-01}: Niepoprawna reactionRate}
\label{subsec:quant-01}

\begin{criticalbox}[title={\bugid{QUANT-01} --- Niepoprawny mianownik reactionRate}]

\textbf{Lokalizacja:} \filepath{src/lib/analysis/quantitative.ts}, linia~588--589

\textbf{Opis:} Współczynnik \texttt{reactionRate} dzieli \texttt{reactionsGiven}
przez \texttt{messagesReceived}. Semantycznie jest to stosunek
\emph{„ile reakcji dała osoba A na wiadomości od~innych"}, co mierzy
\textbf{aktywność reakcyjną} danej osoby, a~nie \textbf{otrzymany engagement}.

\begin{lstlisting}[style=podcode, caption={Obliczanie reactionRate --- aktualny kod}]
reactionRate[name] =
  acc.messagesReceived > 0
    ? acc.reactionsGiven / acc.messagesReceived
    : 0;
\end{lstlisting}

\textbf{Problem:} W~czatach grupowych (3+ osób) \texttt{messagesReceived}
obejmuje wiadomości od~\emph{wszystkich} pozostałych uczestników, podczas
gdy \texttt{reactionsGiven} może dotyczyć reakcji na~wiadomości tylko
wybranych osób. Powoduje to~sztuczne zaniżenie rate w~dużych grupach.

Ponadto, termin \texttt{reactionRate} sugeruje, że~mierzy się \emph{stopień
reagowania na~wiadomości danej osoby} (otrzymane reakcje / wysłane wiadomości),
a~nie aktywność reakcyjną.

\textbf{Wpływ:} Niepoprawne wartości \texttt{reactionRate} propagują się do:
\begin{itemize}
  \item \texttt{engagementBalanceScore()} w~viral scores (compatibility score)
  \item \texttt{engagementScore} w~interest score (waga 20\%)
  \item Porównania między osobami w~UI
\end{itemize}

\textbf{Rekomendacja:}
\begin{lstlisting}[style=podcode, caption={Proponowana poprawka QUANT-01}]
// Osobno: rate dawania i otrzymywania reakcji
reactionGiveRate[name] = acc.totalMessages > 0
  ? acc.reactionsGiven / acc.totalMessages : 0;
reactionReceiveRate[name] = acc.totalMessages > 0
  ? acc.reactionsReceived / acc.totalMessages : 0;
\end{lstlisting}
\end{criticalbox}

% ------------------------------------------------------------
\subsection{\bugid{QUANT-02}: responseSymmetry = 100 przy braku danych}
\label{subsec:quant-02}

\begin{criticalbox}[title={\bugid{QUANT-02} --- Brak danych interpretowany jako idealna kompatybilność}]

\textbf{Lokalizacja:} \filepath{src/lib/analysis/viral-scores.ts}, linia~103--104

\begin{lstlisting}[style=podcode, caption={responseSymmetryScore --- aktualny kod}]
function responseSymmetryScore(
  timing: TimingMetrics, names: string[]
): number {
  // ...
  const maxMed = Math.max(medA, medB);
  if (maxMed === 0) return 100; // BUG: brak danych = perfekcja
  // ...
}
\end{lstlisting}

\textbf{Problem:} Gdy obie mediany czasu odpowiedzi wynoszą 0 (brak danych
o~czasach odpowiedzi), funkcja zwraca 100~--- co jest interpretowane jako
\emph{idealna symetria}. Brak danych nie powinien oznaczać perfekcyjnego
dopasowania.

\textbf{Wpływ:} Zawyżony \texttt{compatibilityScore} dla konwersacji
z~niedostateczną ilością danych o~czasach odpowiedzi. Użytkownik widzi
wysoką kompatybilność, której nie da~się uzasadnić danymi.

\textbf{Rekomendacja:} Zwracać wartość neutralną 50 przy braku danych:
\begin{lstlisting}[style=podcode, caption={Proponowana poprawka QUANT-02}]
if (maxMed === 0) return 50; // neutral when no data
\end{lstlisting}
\end{criticalbox}

% ------------------------------------------------------------
\subsection{\bugid{QUANT-03}: Procent response time w~wrapped-data bez clampowania}
\label{subsec:quant-03}

\begin{criticalbox}[title={\bugid{QUANT-03} --- Procent przekraczający 100\% w~Wrapped Mode}]

\textbf{Lokalizacja:} \filepath{src/lib/analysis/wrapped-data.ts}, linia~230--231

\begin{lstlisting}[style=podcode, caption={Wrapped slide --- response time percent}]
personA: {
  name: nameA,
  value: formatMinutes(rtA),
  percent: rtA <= rtB
    ? 100
    : Math.round((rtB / (rtA || 1)) * 100)
},
personB: {
  name: nameB,
  value: formatMinutes(rtB),
  percent: rtB <= rtA
    ? 100
    : Math.round((rtA / (rtB || 1)) * 100)
},
\end{lstlisting}

\textbf{Problem:} Gdy \texttt{rtB >> rtA}, wartość
\texttt{Math.round((rtB / (rtA || 1)) * 100)} może znacznie przekroczyć
100. Nie ma żadnego clampowania wyniku. Mimo że~szybsza osoba dostaje
\texttt{percent: 100}, procent wolniejszej osoby jest obliczany jako
stosunek odwrotny~--- i~tutaj logika jest poprawna (szybszy/wolniejszy
daje wartość <100). Jednakże, gdyby obie wartości były bardzo zbliżone
do~0, mogą pojawić się wartości >100 przez zaokrąglenia.

\textbf{Wpływ:} Potencjalnie zniekształcony pasek postępu w~Wrapped Mode
na slajdzie ``Kto odpowiada szybciej?''.

\textbf{Rekomendacja:} Dodać \texttt{Math.min(..., 100)} do~obu obliczeń:
\begin{lstlisting}[style=podcode, caption={Proponowana poprawka QUANT-03}]
percent: Math.min(100, Math.round((rtB / (rtA || 1)) * 100))
\end{lstlisting}
\end{criticalbox}

% ------------------------------------------------------------
\subsection{\bugid{QUANT-04}: delusionHolder semantycznie odwrócony}
\label{subsec:quant-04}

\begin{criticalbox}[title={\bugid{QUANT-04} --- delusionHolder przypisany do~osoby z~WYŻSZYM zainteresowaniem}]

\textbf{Lokalizacja:} \filepath{src/lib/analysis/viral-scores.ts}, linie~421--428

\begin{lstlisting}[style=podcode, caption={Obliczanie delusionHolder --- aktualny kod}]
if (interestValues.length >= 2) {
  const sorted = [...interestValues].sort((a, b) => b[1] - a[1]);
  delusionScore = Math.abs(sorted[0][1] - sorted[1][1]);
  delusionHolder = sorted[0][0]; // BUG: osoba z NAJWYZSZYM interest
  if (delusionScore < 5) {
    delusionHolder = undefined;
  }
}
\end{lstlisting}

\textbf{Problem:} Tablica jest sortowana malejąco (\texttt{b[1] - a[1]}),
więc \texttt{sorted[0]} to~osoba z~\textbf{najwyższym} wynikiem Interest Score.
\texttt{delusionHolder} jest przypisywany do~niej.

Semantyka Delusion Score: powinna wskazywać osobę, która~\textbf{przecenia
wzajemność} --- czyli osobę z~\emph{niższym} interest score od~drugiej strony,
która (prawdopodobnie) żyje w~złudzeniu, że~zainteresowanie jest wzajemne.

Osoba z~wyższym interest score jest bardziej zaangażowana \emph{i~świadoma tego}
--- to~nie jest delusion. Delusion to~osoba z~niższym interest, która
nie dostrzega asymetrii.

\textbf{Wpływ:} Odwrócone wyniki na kartach Delusion Score --- użytkownik
widzi ``X żyje w~złudzeniach'', ale to~tak naprawdę osoba Y jest
mniej zaangażowana i~potencjalnie bardziej zdeluzjonowana.

\textbf{Rekomendacja:}
\begin{lstlisting}[style=podcode, caption={Proponowana poprawka QUANT-04}]
delusionHolder = sorted[1][0]; // osoba z NIZSZYM interest score
\end{lstlisting}

\textbf{Uwaga:} Definicja ``delusion'' jest dyskusyjna i~zależy od~kontekstu.
Alternatywna interpretacja: osoba z~wyższym interest jest~``zdeluzjonowana'',
bo nadmiernie inwestuje emocjonalnie. W~takim wypadku obecna logika
byłaby poprawna, ale wymaga wyraźnej dokumentacji decyzji projektowej.
\end{criticalbox}

% ------------------------------------------------------------
\subsection{\bugid{QUANT-05}: reciprocity responseTimeSymmetry = 100 przy braku danych}
\label{subsec:quant-05}

\begin{criticalbox}[title={\bugid{QUANT-05} --- Brak danych RT = perfekcyjna synchronizacja w~ReciprocityIndex}]

\textbf{Lokalizacja:} \filepath{src/lib/analysis/quant/reciprocity.ts}, linie~62--70

\begin{lstlisting}[style=podcode, caption={reciprocity responseTimeSymmetry --- aktualny kod}]
let responseTimeSymmetry = 50;
if (rtA > 0 && rtB > 0) {
  const ratio = Math.min(rtA, rtB) / Math.max(rtA, rtB);
  responseTimeSymmetry = Math.round(ratio * 100);
} else if (rtA === 0 && rtB === 0) {
  responseTimeSymmetry = 50; // poprawnie: neutral
}
\end{lstlisting}

\textbf{Problem:} Na~pierwszy rzut oka kod wygląda poprawnie~--- zwraca 50 gdy
oba RT wynoszą~0. Jednakże brakuje obsługi przypadku, gdy \textbf{tylko
jedno} RT wynosi~0 (np.~osoba A~nigdy nie odpowiadała, ale osoba B~tak).
W~takim wypadku \texttt{responseTimeSymmetry} pozostaje na~wartości
domyślnej 50~--- co nie oddaje skrajnej asymetrii (jedna osoba odpowiada,
druga nie).

\textbf{Powiązanie z~\bugid{QUANT-02}:} Analogiczny wzorzec: brak danych
traktowany nadmiernie optymistycznie. W~\filepath{viral-scores.ts}
to~samo zjawisko daje wynik 100 zamiast 50.

\textbf{Wpływ:} Zawyżony ReciprocityIndex w~konwersacjach, gdzie jedna
osoba ma dane o~response time, a~druga nie.

\textbf{Rekomendacja:}
\begin{lstlisting}[style=podcode, caption={Proponowana poprawka QUANT-05}]
if (rtA > 0 && rtB > 0) {
  const ratio = Math.min(rtA, rtB) / Math.max(rtA, rtB);
  responseTimeSymmetry = Math.round(ratio * 100);
} else if (rtA === 0 && rtB === 0) {
  responseTimeSymmetry = 50; // brak danych - neutral
} else {
  responseTimeSymmetry = 10; // skrajna asymetria
}
\end{lstlisting}
\end{criticalbox}


% ============================================================
\section{Problemy istotne}
\label{sec:quant-moderate}
\index{problemy istotne!silnik ilościowy}

Zidentyfikowano 8~problemów istotnych. Nie są to~błędy logiczne sensu stricto,
ale unjustified magic numbers, nieoptymalne wybory miar statystycznych
i~edge case'y wpływające na~jakość wyników.

% ------------------------------------------------------------
\subsection{\issueid{ENG-01}: Hardcoded slope normalization w~interest score}
\label{subsec:eng-01}

\begin{moderatebox}[title={\issueid{ENG-01} --- Magiczna stała 1200 w~normalizacji response time trend}]

\textbf{Lokalizacja:} \filepath{src/lib/analysis/viral-scores.ts}, linia~197

\begin{lstlisting}[style=podcode, caption={Response time trend --- normalizacja}]
// Normalize: negative slope is good.
// A slope of -60000ms/month is very good.
// Map range: slope <= -60000 -> 100, slope >= 60000 -> 0
const rtScore = clamp(50 - safeDivide(rtSlope, 1200), 0, 100);
\end{lstlisting}

Stała \texttt{1200} wynika z~$60000 / 50 = 1200$. Komentarz wyjaśnia intencję,
ale wartość 60\,000\,ms/miesiąc (1~minuta/miesiąc) jako ``bardzo dobra'' zmiana
jest arbitralna i~nie została zwalidowana empirycznie.

\textbf{Wpływ:} Normalizacja może być zbyt agresywna lub zbyt łagodna
w~zależności od~charakterystyki konwersacji. Brak adaptacji do~baseline
danej rozmowy.
\end{moderatebox}

% ------------------------------------------------------------
\subsection{\issueid{ENG-02}: Mnożnik 25 w~trendzie długości wiadomości}
\label{subsec:eng-02}

\begin{moderatebox}[title={\issueid{ENG-02} --- Nieuzasadniony mnożnik w~message length trend}]

\textbf{Lokalizacja:} \filepath{src/lib/analysis/viral-scores.ts}, linia~206

\begin{lstlisting}[style=podcode, caption={Message length trend score}]
// Positive slope = longer messages = more engaged
// Map: slope of +2 words/month -> 100, slope of -2 -> 0
const mlScore = clamp(50 + mlSlope * 25, 0, 100);
\end{lstlisting}

Mnożnik \texttt{25} oznacza, że~zmiana o~$\pm2$ słowa/miesiąc daje wynik 0 lub 100.
Nie ma uzasadnienia, dlaczego 2~słowa/miesiąc to~granica. W~krótkich konwersacjach
(średnio 3--5~słów) zmiana o~2~słowa to~40--67\% zmiany, ale w~dłuższych
(20+~słów) to~zaledwie 10\%.

\textbf{Rekomendacja:} Normalizować slope względem baseline'owej średniej
długości wiadomości danej osoby.
\end{moderatebox}

% ------------------------------------------------------------
\subsection{\issueid{ENG-03}: Mnożnik 500 w~engagement balance}
\label{subsec:eng-03}

\begin{moderatebox}[title={\issueid{ENG-03} --- Zbyt ekstremalny mnożnik engagement balance}]

\textbf{Lokalizacja:} \filepath{src/lib/analysis/viral-scores.ts}, linia~144

\begin{lstlisting}[style=podcode, caption={engagementBalanceScore --- mnożnik 500}]
return clamp(100 - Math.abs(rateA - rateB) * 500, 0, 100);
\end{lstlisting}

Mnożnik 500 oznacza, że~różnica \texttt{reactionRate} wynosząca zaledwie 0{,}2
(20~punktów procentowych) daje wynik 0. Typowe wartości \texttt{reactionRate}
to~0{,}01--0{,}10, więc nawet niewielka różnica absolutna może spowodować
zerowy wynik engagement balance.

\textbf{Wpływ:} Compatibility score jest nadmiernie wrażliwy na~różnice
w~częstości reakcji. Para, gdzie osoba A~reaguje na~5\% wiadomości,
a~osoba B~na~7\%, otrzyma engagement balance = 90, ale para 5\% vs.~25\%
otrzyma~0.

\textbf{Rekomendacja:} Użyć normalizacji opartej na~stosunku
(min/max) zamiast różnicy absolutnej z~mnożnikiem.
\end{moderatebox}

% ------------------------------------------------------------
\subsection{\issueid{ENG-04}: Early Bird badge liczy wartości bezwzględne}
\label{subsec:eng-04}

\begin{moderatebox}[title={\issueid{ENG-04} --- Badge Early Bird faworyzuje osoby z~większą liczbą wiadomości}]

\textbf{Lokalizacja:} \filepath{src/lib/analysis/badges.ts}, linie~182--206

\begin{lstlisting}[style=podcode, caption={Early Bird --- bezwzględna liczba wiadomości}]
const earlyBirdCount: Record<string, number> = {};
for (const name of names) {
  let count = 0;
  const matrix = heatmap.perPerson[name];
  if (matrix) {
    for (let day = 0; day < 7; day++) {
      for (let hour = 0; hour < 8; hour++) {
        count += matrix[day][hour];
      }
    }
  }
  earlyBirdCount[name] = count;
}
\end{lstlisting}

Badge liczy bezwzględną liczbę wiadomości przed 8:00. Osoba wysyłająca
10\,000 wiadomości (z~czego 200~przed 8:00 = 2\%) wygra nad osobą
wysyłającą 1\,000 wiadomości (z~czego 150~przed 8:00 = 15\%).

\textbf{Kontrast:} Night Owl poprawnie używa procentu:
\begin{lstlisting}[style=podcode, caption={Night Owl --- poprawny procent}]
lateNightPct[name] = total > 0 ? (lateNight / total) * 100 : 0;
\end{lstlisting}

\textbf{Rekomendacja:} Użyć analogicznego podejścia procentowego jak Night Owl.
\end{moderatebox}

% ------------------------------------------------------------
\subsection{\issueid{ENG-05}: Próg unikalności catchphrase za niski}
\label{subsec:eng-05}

\begin{moderatebox}[title={\issueid{ENG-05} --- Catchphrase uniqueness threshold 0{,}6 za niski}]

\textbf{Lokalizacja:} \filepath{src/lib/analysis/catchphrases.ts}, linia~144

\begin{lstlisting}[style=podcode, caption={Catchphrase uniqueness threshold}]
if (uniqueness < 0.6) continue;
\end{lstlisting}

Próg 0{,}6 oznacza, że~fraza, która~jest używana w~60\% przez osobę A
i~w~40\% przez osobę B, nadal kwalifikuje się jako ``catchphrase'' osoby~A.
Tak niska unikalność podważa wiarygodność etykiety~--- fraza używana
niemal równo przez obu rozmówców nie jest charakterystycznym zwrotem
żadnego z~nich.

\textbf{Rekomendacja:} Podnieść próg do~0{,}75 lub wyżej, aby catchphrases
były naprawdę unikalne dla danej osoby.
\end{moderatebox}

% ------------------------------------------------------------
\subsection{\issueid{ENG-06}: Trend message length używa mean zamiast median}
\label{subsec:eng-06}

\begin{moderatebox}[title={\issueid{ENG-06} --- Średnia arytmetyczna wrażliwa na~outliers w~trendzie długości}]

\textbf{Lokalizacja:} \filepath{src/lib/analysis/quant/trends.ts}, linie~43--50

\begin{lstlisting}[style=podcode, caption={Message length trend --- mean}]
// Message length trend: monthly average word count per person
const messageLengthTrend = sortedMonths.map((month) => {
  const pp: Record<string, number> = {};
  for (const [name, acc] of accumulators) {
    const words = acc.monthlyWordCounts.get(month);
    if (words && words.length > 0) {
      pp[name] = words.reduce((a, b) => a + b, 0) / words.length;
    } else {
      pp[name] = 0;
    }
  }
  return { month, perPerson: pp };
});
\end{lstlisting}

\textbf{Problem:} Trend response time poprawnie używa \textbf{mediany}
(z~filtrowaniem outliers), ale trend długości wiadomości używa
\textbf{średniej arytmetycznej}. Jedna bardzo długa wiadomość (np.~wklejony
tekst, lista zakupów) może drastycznie zawyżyć średnią w~danym miesiącu
i~zniekształcić trend.

\textbf{Rekomendacja:} Użyć mediany, analogicznie do~response time trend.
\end{moderatebox}

% ------------------------------------------------------------
\subsection{\issueid{ENG-07}: Arbitralny próg detekcji burstów}
\label{subsec:eng-07}

\begin{moderatebox}[title={\issueid{ENG-07} --- Próg 3$\times$ średnia krocząca bez walidacji}]

\textbf{Lokalizacja:} \filepath{src/lib/analysis/quant/bursts.ts}, linia~49

\begin{lstlisting}[style=podcode, caption={Próg detekcji burstu}]
if (dayValues[i].count > 3 * rollingAvg && rollingAvg > 0) {
  burstDays.push(dayValues[i]);
}
\end{lstlisting}

Próg $3\times$ 7-dniowej średniej kroczącej jest powszechnie stosowany,
ale arbitralny. Nie uwzględnia:
\begin{itemize}
  \item Wariancji lokalnej --- w~aktywnej konwersacji z~dużą wariancją
    dzienną, $3\times$ średnia może być standardowym dniem.
  \item Dla konwersacji z~niską aktywnością (np.~2~wiadomości dziennie),
    6~wiadomości w~ciągu dnia to~burst, choć merytorycznie to~nic nadzwyczajnego.
\end{itemize}

\textbf{Rekomendacja:} Rozważyć próg oparty na~odchyleniu standardowym
($\mu + 2\sigma$) lub adaptywny próg uwzględniający skalę konwersacji.
\end{moderatebox}

% ------------------------------------------------------------
\subsection{\issueid{ENG-08}: Peak hour window~--- edge case godziny 23}
\label{subsec:eng-08}

\begin{moderatebox}[title={\issueid{ENG-08} --- Okno Best Time to Text nie owijaj się wokół północy}]

\textbf{Lokalizacja:} \filepath{src/lib/analysis/catchphrases.ts}, linie~215--217

\begin{lstlisting}[style=podcode, caption={Best Time to Text --- window}]
const windowStart = bestHour > 0 ? bestHour : 0;
const windowEnd = Math.min(windowStart + 2, 24);
\end{lstlisting}

Gdy \texttt{bestHour = 23}, okno wynosi 23:00--24:00 (1~godzina zamiast 2).
Ponadto warunek \texttt{bestHour > 0} powoduje, że~godzina~0 nie jest
traktowana inaczej niż pozostałe, ale \texttt{windowStart} jest identyczny
z~\texttt{bestHour} w~prawie wszystkich przypadkach.

\textbf{Rekomendacja:} Okno powinno owijać się wokół północy:
\begin{lstlisting}[style=podcode, caption={Proponowana poprawka ENG-08}]
const windowStart = bestHour;
const windowEnd = (windowStart + 2) % 24;
const windowStr = windowEnd < windowStart
  ? `${formatHour(windowStart)}-${formatHour(windowEnd)} (nast. dzien)`
  : `${formatHour(windowStart)}-${formatHour(windowEnd)}`;
\end{lstlisting}
\end{moderatebox}


% ============================================================
\section{Problemy drobne}
\label{sec:quant-minor}
\index{problemy drobne!silnik ilościowy}

Zidentyfikowano 5~problemów drobnych o~niskim wpływie na~użytkownika końcowego:

\begin{enumerate}[label=\textcolor{PodTextSecondary}{\textbf{\arabic*.}}]

  \item \minor{Shortest message inicjalizowane na~Infinity}
    \hfill \filepath{quant/types.ts:60}\\
    \texttt{shortestMessage.length = Infinity} jest poprawnie obsłużone
    w~post-processingu (zamieniane na~0), ale wartość \texttt{Infinity}
    w~akumulatorze jest nieintuicyjna i~może powodować problemy, jeśli
    zostanie przypadkowo użyta przed post-processingiem.

  \item \minor{Discord reaction rate fallback semantycznie odwrócony}
    \hfill \filepath{viral-scores.ts:134--141}\\
    Gdy \texttt{reactionRate} = 0 dla obu osób (Discord), fallback
    używa \texttt{mentionRate + replyRate}. Mnożniki 500 i~200 są~takie
    same jak dla standardowego engagement balance, ale mentions i~replies
    mają inną skalę niż reakcje~--- mnożniki powinny być dostosowane.

  \item \minor{Network density ignoruje wagi krawędzi}
    \hfill \filepath{network.ts:96--97}\\
    Density jest obliczane jako proporcja krawędzi o~wadze > 0 do~wszystkich
    możliwych krawędzi. Nie uwzględnia to~wag~--- para z~1~interakcją
    ma~taki sam wpływ jak para z~1\,000~interakcji.

  \item \minor{Regresja liniowa zakłada równoodległy axis X}
    \hfill \filepath{constants.ts:50}\\
    Funkcja \tsfunc{linearRegressionSlope()} traktuje indeksy tablicy
    jako wartości X (0, 1, 2, ...). Gdy brakuje miesięcy w~danych (np.~przerwa
    w~konwersacji), regresja zakłada ciągłość, co~zniekształca nachylenie.

  \item \minor{Wrapped mode ``book equivalents'' — 50k słów/książka}
    \hfill \filepath{wrapped-data.ts:175}\\
    \texttt{const books = (totalWords / 50\_000).toFixed(1)} --- typowa
    książka ma~70\,000--100\,000~słów. Wartość 50\,000 zawyża liczbę
    ``książek'' o~40--100\%.
\end{enumerate}


% ============================================================
\section{Analiza metryk wiralnych}
\label{sec:quant-viral}
\index{metryki wiralne}

Moduł \filepath{viral-scores.ts} oblicza cztery główne metryki
przeznaczone do~udostępniania w~mediach społecznościowych. Każda
została poddana szczegółowej analizie psychometrycznej i~matematycznej.

% ------------------------------------------------------------
\subsection{Compatibility Score (0--100)}
\label{subsec:viral-compat}
\index{Compatibility Score}

Wynik kompatybilności jest średnią arytmetyczną pięciu pod-wyników:

\begin{equation}
  \label{eq:compatibility}
  \text{compatibilityScore} = \frac{1}{5}\sum_{i=1}^{5} s_i
\end{equation}

\noindent gdzie:

\begin{tabularx}{\textwidth}{l l X}
  \toprule
  \textbf{Pod-wynik $s_i$} & \textbf{Zakres} & \textbf{Opis} \\
  \midrule
  Activity Overlap & 0--100 & Nakładanie się dystrybucji godzinowej aktywności
    (współczynnik Szymkiewicza-Simpsona\index{Szymkiewicz-Simpson}) \\
  Response Symmetry & 0--100 & Symetria median czasu odpowiedzi \\
  Message Balance & 0--100 & Równomierność podziału wiadomości \\
  Engagement Balance & 0--100 & Symetria rate'ów reakcji \\
  Length Match & 0--100 & Podobieństwo średnich długości wiadomości \\
  \bottomrule
\end{tabularx}

\vspace{6pt}

\subsubsection{Activity Overlap}

Implementacja poprawnie oblicza nakładanie się rozkładów godzinowych:

\begin{equation}
  \text{overlap} = \sum_{h=0}^{23} \min\!\left(\frac{A_h}{\sum A}, \frac{B_h}{\sum B}\right) \cdot 100
\end{equation}

Jest to~wariant współczynnika Szymkiewicza-Simpsona na~rozkładach dyskretnych.
\scorehi{Poprawna implementacja.}

\subsubsection{Response Symmetry}

\begin{equation}
  \text{responseSymmetry} = 100 - \frac{|\text{med}_A - \text{med}_B|}{\max(\text{med}_A, \text{med}_B)} \cdot 100
\end{equation}

Poprawna formuła, ale z~bugiem \bugid{QUANT-02} (zwraca 100 przy braku danych).

\subsubsection{Ocena psychologiczna}

\begin{warningbox}[title={Ważenie równomierne jest naiwne}]
Traktowanie wszystkich 5~pod-wyników z~jednakową wagą ($\frac{1}{5}$) nie
ma uzasadnienia psychologicznego. Response Symmetry i~Message Balance mają
silniejszy związek z~postrzeganą jakością relacji niż Length Match.

\textbf{Brakujące wymiary:}
\begin{itemize}
  \item Analiza sentymentu / tonu emocjonalnego
  \item Wspólne tematy (topic overlap)
  \item Wzajemność pytań i~odpowiedzi
  \item Spójność w~czasie (stabilność kompatybilności)
\end{itemize}

\textbf{Werdykt:} \scorelo{Psychologicznie słaby.} Mierzy podobieństwo
zachowań, nie kompatybilność relacyjną.
\end{warningbox}

% ------------------------------------------------------------
\subsection{Interest Score (0--100)}
\label{subsec:viral-interest}
\index{Interest Score}

Interest Score mierzy ``zainteresowanie'' danej osoby rozmową
za~pomocą 6~komponentów z~wagami:

\begin{equation}
  \label{eq:interest}
  \text{interestScore} = \sum_{j=1}^{6} w_j \cdot c_j
\end{equation}

\begin{table}[H]
  \caption{Komponenty Interest Score z~wagami}
  \label{tab:interest-components}
  \centering
  \begin{tabularx}{\textwidth}{l c X l}
    \toprule
    \textbf{Komponent $c_j$} & \textbf{Waga $w_j$} & \textbf{Co mierzy} & \textbf{Normalizacja} \\
    \midrule
    Initiation ratio & 25\% & Odsetek rozpoczętych rozmów & Liniowa \\
    RT trend & 20\% & Nachylenie trendu czasu odpowiedzi & $50 - \frac{\text{slope}}{1200}$ \\
    Msg length trend & 15\% & Nachylenie trendu długości wiad. & $50 + \text{slope} \cdot 25$ \\
    Engagement & 20\% & Częstość reakcji / mentions & $\text{rate} \cdot 500$ \\
    Double texting & 10\% & Częstość double textów & $\frac{\text{dt} \cdot 1000}{\text{total}} \cdot 2$ \\
    Late night ratio & 10\% & Odsetek wiadomości nocnych & $\frac{\text{late}}{\text{total}} \cdot 1000$ \\
    \bottomrule
  \end{tabularx}
\end{table}

\subsubsection{Problemy z~komponentami}

\begin{itemize}
  \item \textbf{Late night jako sygnał zainteresowania} jest kulturowo
    stronniczy. W~wielu kulturach i~grupach wiekowych pisanie w~nocy
    wynika z~trybu życia (praca zmianowa, studenci), nie z~romantycznego
    zainteresowania.
  \item \textbf{Double texting} jako wskaźnik zainteresowania: w~literaturze
    z~zakresu komunikacji cyfrowej double texting koreluje zarówno
    z~zaangażowaniem, jak i~z~niepokojem (anxiety). Nie jest jednoznaczny.
  \item \textbf{Behavioral proxies $\neq$ romantic interest}: wszystkie
    komponenty mierzą zachowania komunikacyjne, które mogą wynikać
    z~wielu przyczyn (zawodowych, towarzyskich, lękowych), nie~tylko
    z~romantycznego zainteresowania.
\end{itemize}

\textbf{Werdykt:} \scoremed{Umiarkowana trafność.} Przydatny jako
wskaźnik asymetrii zaangażowania, ale etykieta ``Interest Score''
sugeruje romantyczne zainteresowanie, którego metryki nie mierzą.

% ------------------------------------------------------------
\subsection{Ghost Risk (0--100)}
\label{subsec:viral-ghost}
\index{Ghost Risk}

Ghost Risk mierzy prawdopodobieństwo ``ghostingu'' na~podstawie
4~czynników:

\begin{equation}
  \label{eq:ghost}
  \text{ghostRisk} = 0{,}30 \cdot r_{\text{RT}} + 0{,}25 \cdot r_{\text{ML}} + 0{,}25 \cdot r_{\text{init}} + 0{,}20 \cdot r_{\text{vol}}
\end{equation}

\begin{table}[H]
  \caption{Czynniki Ghost Risk}
  \label{tab:ghost-factors}
  \centering
  \begin{tabularx}{\textwidth}{l c X}
    \toprule
    \textbf{Czynnik} & \textbf{Waga} & \textbf{Opis} \\
    \midrule
    RT increasing & 30\% & Rosnący czas odpowiedzi (ostatnie 3 mies. vs wcześniej) \\
    Msg length decreasing & 25\% & Malejąca długość wiadomości \\
    Initiation decreasing & 25\% & Malejąca częstość inicjowania rozmów \\
    Volume declining & 20\% & Malejący wolumen wiadomości \\
    \bottomrule
  \end{tabularx}
\end{table}

\subsubsection{Mocne strony}
\begin{itemize}
  \item Wymaga minimum 6~miesięcy danych (\texttt{months.length >= 3}
    z~podziałem recent/earlier) --- rozsądne minimum.
  \item Porównanie recent (3~mies.) vs earlier to~sensowna heurystyka.
  \item 4~czynniki mają logiczny związek z~wycofywaniem się z~konwersacji.
\end{itemize}

\subsubsection{Słabości}
\begin{itemize}
  \item Nie rozróżnia między ghostingiem a~zewnętrznymi czynnikami
    (stres zawodowy, choroba, wakacje).
  \item Brak sezonowości~--- spadek aktywności latem może nie oznaczać
    ghostingu.
  \item Wymaga 6+ miesięcy, co~eliminuje wiele par.
\end{itemize}

\textbf{Werdykt:} \scoremed{Umiarkowana trafność.} Najlepszy z~czterech
viral scores pod~względem konstrukcji, ale z~ważnymi confounders.

% ------------------------------------------------------------
\subsection{Delusion Score}
\label{subsec:viral-delusion}
\index{Delusion Score}

\begin{equation}
  \label{eq:delusion}
  \text{delusionScore} = |\text{interest}_{\text{top}} - \text{interest}_{\text{second}}|
\end{equation}

Delusion Score jest prostą różnicą bezwzględną dwóch najwyższych
Interest Scores. Próg istotności wynosi 5~punktów.

\begin{warningbox}[title={Metryka wymaga redesignu}]
\begin{enumerate}
  \item \textbf{Semantycznie odwrócony holder} (\bugid{QUANT-04})~---
    przypisuje delusion do~osoby z~wyższym, nie niższym interestem.
  \item \textbf{Arbitralny próg 5~punktów}~--- brak uzasadnienia
    statystycznego. Różnica 5~vs~6~punktów Interest Score (skala 0--100)
    mieści się w~szumie pomiarowym.
  \item \textbf{Brak normalizacji}~--- różnica 5~przy wynikach 90~vs~85
    ma~inne znaczenie niż przy 55~vs~50.
  \item \textbf{Metryka pochodna metryki}~--- opiera się na~Interest Score,
    który sam ma ograniczoną trafność. Propagacja błędów.
\end{enumerate}

\textbf{Werdykt:} \scorelo{Wymaga redesignu.} W~obecnej formie
generuje więcej szumu niż sygnału.
\end{warningbox}


% ============================================================
\section{Indeks Wzajemności (ReciprocityIndex)}
\label{sec:quant-reciprocity}
\index{ReciprocityIndex}

Moduł \filepath{quant/reciprocity.ts} oblicza 4-wymiarowy indeks
wzajemności konwersacji.

\subsection{Wymiary i~formuły}

\begin{enumerate}
  \item \textbf{Message Balance:}
  \begin{equation}
    \text{messageBalance} = 100 \cdot (1 - 2\,|\text{ratioA} - 0{,}5|)
  \end{equation}
  Gdzie \text{ratioA} = $\frac{\text{msg}_A}{\text{msg}_A + \text{msg}_B}$.
  Wynik 100 przy idealnym podziale 50/50, 0 przy 100/0.

  \item \textbf{Initiation Balance:}
  \begin{equation}
    \text{initiationBalance} = 100 \cdot \left(1 - 2\,\left|\frac{\text{init}_A}{\text{init}_A + \text{init}_B} - 0{,}5\right|\right)
  \end{equation}

  \item \textbf{Response Time Symmetry:}
  \begin{equation}
    \text{responseTimeSymmetry} = \frac{\min(\text{rt}_A, \text{rt}_B)}{\max(\text{rt}_A, \text{rt}_B)} \cdot 100
  \end{equation}
  Stosunek min/max, 100 przy identycznych czasach. Bug~\bugid{QUANT-05}
  dotyczy obsługi zerowych wartości.

  \item \textbf{Reaction Balance:}
  \begin{equation}
    \text{reactionBalance} = 100 \cdot \left(1 - 2\,\left|\frac{\text{react}_A}{\text{react}_A + \text{react}_B} - 0{,}5\right|\right)
  \end{equation}
  Z~Discord fallback na~\texttt{mentionsReceived + repliesReceived}.
\end{enumerate}

\subsection{Wynik ogólny}

\begin{equation}
  \text{overall} = \frac{1}{4}\sum_{d=1}^{4} d_i
\end{equation}

Równe wagi dla wszystkich 4~wymiarów.

\subsection{Ocena}

\begin{strengthbox}[title={Mocne strony ReciprocityIndex}]
\begin{itemize}
  \item Prostota i~interpretowalność~--- każdy pod-wynik ma~jasne znaczenie.
  \item Formuły $1 - 2|x - 0{,}5|$ są~eleganckie i~poprawne matematycznie.
  \item Discord fallback z~mentions/replies to~rozsądny proxy.
\end{itemize}
\end{strengthbox}

\begin{moderatebox}[title={Słabości ReciprocityIndex}]
\begin{itemize}
  \item Równe ważenie 4~wymiarów~--- Message Balance i~Response Time Symmetry
    mają prawdopodobnie większy wpływ na~percypowaną wzajemność niż
    Reaction Balance.
  \item Brak wymiaru ``inicjowania tematów'' (kto wprowadza nowe wątki).
  \item Skala 0--100 z~równym ważeniem: wynik 75 może oznaczać np.~100/100/100/0
    (perfekcja w~3 wymiarach, katastrofa w~1)~--- to~nie jest ``dobra'' wzajemność.
\end{itemize}
\end{moderatebox}


% ============================================================
\section{System odznak}
\label{sec:quant-badges}
\index{odznaki}

Moduł \filepath{badges.ts} przyznaje 15~odznak behawioralnych.
Poniższa tabela podsumowuje logikę, progi i~zidentyfikowane problemy:

\begin{table}[H]
  \caption{System 15~odznak behawioralnych}
  \label{tab:badges}
  \centering
  \small
  \begin{tabularx}{\textwidth}{l l l X}
    \toprule
    \textbf{ID} & \textbf{Nazwa} & \textbf{Kryterium} & \textbf{Uwagi} \\
    \midrule
    night-owl & Nocny Marek & Max \% wiad. 22--4 & \scorehi{Poprawnie (procent)} \\
    early-bird & Ranny Ptaszek & Max wiad. przed 8:00 & \scorelo{\issueid{ENG-04}: bezwzgl. nie \%} \\
    ghost-champion & Ghosting Champion & Ostatni przed ciszą & Poprawne \\
    double-texter & Double Texter & Max double textów & Bezwzgl. wartość, ok \\
    novelist & Powieściopisarz & Max śr. dł. wiad. & Poprawne \\
    speed-demon & Speed Demon & Min mediana RT & Poprawne \\
    emoji-monarch & Emoji King/Queen & Max emoji/wiad. & \scorehi{Poprawnie (rate)} \\
    initiator & Inicjator & Max inicjacji & Bezwzgl., ale z~\% w~evidence \\
    heart-bomber & Heart Bomber & Max reakcji serduszkowych & Poprawne, skip Discord \\
    link-lord & Link Lord & Max linków & Bezwzględne, ok \\
    streak-master & Streak Master & Max dni z~rzędu & Poprawne \\
    question-master & Detektyw & Max pytań & Bezwzgl., ok \\
    mention-magnet & Magnes na @ & Max mentions recv >5 & Discord only, próg ok \\
    reply-king & Król Odpowiedzi & Max replies sent >10 & Discord only, próg ok \\
    edit-lord & Perfekcjonista & Max edits >5 & Discord only, próg ok \\
    \bottomrule
  \end{tabularx}
\end{table}

\vspace{4pt}

Ogólna ocena systemu odznak: \scoremed{7/10}. Dobrze zaimplementowany,
zabawny i~angażujący. Główny problem to~\issueid{ENG-04} (Early Bird
bezwzględnie zamiast procentowo) oraz brak normalizacji niektórych
wartości do~rozmiaru konwersacji.


% ============================================================
\section{Wykrywanie burstów i~trendy}
\label{sec:quant-bursts-trends}
\index{burst detection}
\index{trendy}

% ------------------------------------------------------------
\subsection{Detekcja burstów}
\label{subsec:quant-bursts}

Algorytm w~\filepath{quant/bursts.ts}:

\begin{enumerate}
  \item Oblicza 7-dniową średnią kroczącą (rolling average) dla dziennych
    liczników wiadomości.
  \item Dla pierwszych 7~dni używa średniej globalnej jako baseline.
  \item Dni, w~których liczba wiadomości przekracza $3\times$ rolling average,
    są~oznaczane jako burst.
  \item Kolejne dni burstowe są~łączone w~okresy burstowe.
\end{enumerate}

\begin{moderatebox}[title={Ocena algorytmu detekcji burstów}]
\textbf{Zalety:}
\begin{itemize}
  \item Prosta, czytelna implementacja.
  \item Łączenie kolejnych dni burstowych w~okresy~--- trafne.
  \item Wymaga minimum 8~dni danych (\texttt{sortedDays.length < 8}).
\end{itemize}

\textbf{Wady} (\issueid{ENG-07}):
\begin{itemize}
  \item Próg $3\times$ jest arbitralny i~nie uwzględnia lokalnej wariancji.
  \item Fallback na~średnią globalną dla pierwszych 7~dni może generować
    fałszywe alarmy na~początku konwersacji.
  \item Brak walidacji minimalnej liczby wiadomości dziennych~---
    burst przy baseline 1~wiad./dzień to~zaledwie 4~wiadomości.
\end{itemize}
\end{moderatebox}

% ------------------------------------------------------------
\subsection{System trendów}
\label{subsec:quant-trends}

Moduł \filepath{quant/trends.ts} oblicza 3~serie czasowe (miesięczne):

\begin{table}[H]
  \caption{Trendy miesięczne~--- miary statystyczne}
  \label{tab:trends}
  \centering
  \begin{tabularx}{\textwidth}{l l l X}
    \toprule
    \textbf{Trend} & \textbf{Miara} & \textbf{Ocena} & \textbf{Komentarz} \\
    \midrule
    Response time & Mediana (z~filtrem outliers) & \scorehi{Poprawna} & Robustna miara \\
    Message length & Średnia arytmetyczna & \scorelo{\issueid{ENG-06}} & Wrażliwa na~outliers \\
    Initiation count & Wartość bezwzgl./miesiąc & \scoremed{Akceptowalna} & Powinna normalizować do~\% \\
    \bottomrule
  \end{tabularx}
\end{table}

\textbf{Regresja liniowa} (\filepath{constants.ts}):
Funkcja \tsfunc{linearRegressionSlope()} jest poprawna matematycznie,
ale zakłada równoodległy axis X (problem drobny nr~4). W~praktyce
brakujące miesiące są~pomijane, a~nie interpolowane, co~zniekształca
nachylenie w~konwersacjach z~przerwami.


% ============================================================
\section{Metryki sieci (NetworkMetrics)}
\label{sec:quant-network}
\index{NetworkMetrics}
\index{centralność}

Moduł \filepath{network.ts} buduje graf interakcji dla czatów grupowych.
\index{graf interakcji}

\subsection{Konstrukcja macierzy interakcji}

Krawędzie budowane są~z~sekwencyjnych wiadomości:
jeśli osoba~A~wysyła wiadomość, a~następna (w~tej samej sesji)
pochodzi od~osoby~B, to~tworzona jest krawędź $A \to B$.
Graf jest następnie konwertowany na~nieskierowany (obie kierunki sumowane).

\begin{lstlisting}[style=podcode, caption={Budowanie krawędzi grafu interakcji}]
for (let i = 1; i < messages.length; i++) {
  const prev = messages[i - 1];
  const curr = messages[i];
  if (prev.sender === curr.sender) continue;
  if (curr.timestamp - prev.timestamp > SESSION_GAP_MS) continue;
  interactions[prev.sender][curr.sender]++;
}
\end{lstlisting}

\subsection{Metryki}

\begin{itemize}
  \item \textbf{Degree centrality:} $\frac{\text{unikalne połączenia}}{n - 1}$
    gdzie $n$ = liczba uczestników.
  \item \textbf{Density:} $\frac{\text{krawędzie o~wadze} > 0}{\binom{n}{2}}$
  \item \textbf{Most connected:} Osoba z~najwyższą centrality (tiebreak: msg count).
\end{itemize}

\subsection{Ocena}

\begin{moderatebox}[title={Problemy z~metrykami sieciowymi}]
\begin{itemize}
  \item \minor{Density ignoruje wagi krawędzi}~--- para z~1~interakcją
    liczy się tak samo jak para z~500. Prawdziwa gęstość sieci
    powinna uwzględniać siłę połączeń.
  \item \textbf{Brak weighted centrality}~--- degree centrality
    liczy tylko \emph{czy} połączenie istnieje, nie \emph{jak silne} jest.
    Weighted degree centrality ($\sum_j w_{ij}$) byłaby znacznie
    bardziej informatywna.
  \item \textbf{Brak clustering coefficient}~--- nie mierzy się,
    czy~osoby tworzą podgrupy (cliki).
  \item \textbf{Brak betweenness centrality}~--- nie wiadomo, kto
    jest ``łącznikiem'' między podgrupami.
\end{itemize}
\end{moderatebox}


% ============================================================
\section{CPS --- Communication Pattern Screening}
\label{sec:quant-cps}
\index{CPS}
\index{Communication Pattern Screening}

Moduł \filepath{communication-patterns.ts} definiuje 63~pytania
w~10~wzorcach komunikacyjnych. Jest to~\textbf{najsilniejszy
komponent psychologiczny} silnika ilościowego, choć technicznie
odpowiedzi generuje AI (Gemini), nie algorytmy kwantytatywne.

\subsection{Architektura}

\begin{tabularx}{\textwidth}{l c c X}
  \toprule
  \textbf{Wzorzec} & \textbf{Pytań} & \textbf{Próg} & \textbf{Opis} \\
  \midrule
  Unikanie bliskości & 6 & 4 & Dystans emocjonalny \\
  Nadmierna zależność & 7 & 4 & Lęk przed autonomią \\
  Kontrola i~perfekcjonizm & 6 & 4 & Sztywne standardy \\
  Podejrzliwość i~nieufność & 7 & 4 & Szukanie ukrytych znaczeń \\
  Egocentryzm komunikacyjny & 6 & 4 & Skupienie na~sobie \\
  Intensywność emocjonalna & 7 & 4 & Silne reakcje, splitting \\
  Dramatyzacja & 6 & 4 & Teatralność, hiperbole \\
  Manipulacja i~brak empatii & 6 & 3 & Gaslighting, guilt-tripping \\
  Emocjonalny dystans & 6 & 4 & Obojętność, brak ciepła \\
  Pasywna agresja & 6 & 3 & Sarkazm, milczenie \\
  \bottomrule
\end{tabularx}

\subsection{Wymagania}

\begin{itemize}
  \item Minimum 2\,000 wiadomości
  \item Minimum 6~miesięcy danych
  \item Ukończony co~najmniej Pass~1 analizy AI
\end{itemize}

\subsection{Logika obliczania ryzyka}

\begin{lstlisting}[style=podcode, caption={getOverallRiskLevel --- logika OR}]
function getOverallRiskLevel(results: Record<string, CPSPatternResult>) {
  const thresholdMet = Object.values(results)
    .filter((r) => r.meetsThreshold).length;
  const highPercentage = Object.values(results)
    .filter((r) => r.percentage >= 75).length;

  if (thresholdMet >= 2 || highPercentage >= 3) return 'wysoki';
  if (thresholdMet === 1 || highPercentage >= 2) return 'podwyzszony';
  if (highPercentage >= 1)                       return 'umiarkowany';
  return 'niski';
}
\end{lstlisting}

\subsection{Ocena CPS}

\begin{strengthbox}[title={Mocne strony CPS}]
\begin{itemize}
  \item 63~pytania pokrywają szerokie spektrum wzorców komunikacyjnych.
  \item Każde pytanie ma~\texttt{messageSignals}~--- konkretne wskazówki
    dla AI, czego szukać w~wiadomościach.
  \item Rozsądne wymagania minimalne (2\,000~wiadomości, 6~miesięcy).
  \item Jasny disclaimer: ``NIE stanowi diagnozy psychologicznej''.
  \item Polskojęzyczne pytania i~rekomendacje.
\end{itemize}
\end{strengthbox}

\begin{moderatebox}[title={Problemy CPS}]
\begin{itemize}
  \item \textbf{Logika OR w~poziomach ryzyka:} \texttt{thresholdMet >= 2 || highPercentage >= 3}
    łączy dwa różne kryteria operatorem OR. Można jednocześnie mieć
    0~progów przekroczonych, ale 3~wzorce na~75\%~--- to~daje ``wysoki''
    ryzyko, choć żaden wzorzec nie przekroczył progu klinicznego.
  \item \textbf{Brak wagi na~ważność wzorca:} Manipulacja i~brak empatii
    (próg 3) ma~taki sam wpływ na~ogólny wynik jak Dramatyzacja (próg 4),
    mimo że~psychologicznie manipulacja jest poważniejsza.
  \item \textbf{Brak interakcji między wzorcami:} Kombinacja np.~intensywności
    emocjonalnej + manipulacji ma~inną dynamikę niż suma obu oddzielnie.
\end{itemize}
\end{moderatebox}


% ============================================================
\section{Brakujące metryki}
\label{sec:quant-missing}
\index{brakujące metryki}

\begin{featurebox}[title={Metryki nieobecne w~silniku ilościowym}]
Następujące metryki zostały zidentyfikowane jako wartościowe uzupełnienie
istniejącego zestawu:

\begin{enumerate}
  \item \textbf{Analiza sentymentu / emocji}~--- nawet prosty lexicon-based
    sentiment (bez AI) mógłby wzbogacić Compatibility Score i~Interest Score.
    Biblioteki takie jak AFINN czy Sentimentr mają polskie warianty.
    \index{sentiment analysis}

  \item \textbf{Topic modeling / wspólne tematy}~--- TF-IDF na~bigramach
    z~podziałem na~osoby pozwoliłoby mierzyć overlap tematyczny.
    To~naturalny komponent kompatybilności.
    \index{topic modeling}

  \item \textbf{Wzorce konfliktowe}~--- detekcja sekwencji: eskalacja
    (rosnąca intensywność), stonewalling (nagłe milczenie po~wymianie),
    repair attempts (przeprosiny, pojednanie). Kluczowe w~analizie
    dynamiki relacyjnej (model Gottmana).
    \index{wzorce konfliktowe}

  \item \textbf{Progresja wrażliwości / intymności}~--- czy~osoby dzielą
    się coraz bardziej osobistymi informacjami? Leksykon intymności
    (skala od~logistyki po~głębokie emocje) mógłby mierzyć ten wymiar.
    \index{progresja intymności}

  \item \textbf{Weighted network centrality}~--- obecna degree centrality
    ignoruje wagi krawędzi. Weighted degree, betweenness i~clustering
    coefficient wzbogaciłyby analizę grup.
    \index{centralność ważona}

  \item \textbf{Normalizacja trendów inicjacji}~--- obecny trend liczy
    bezwzględne inicjacje/miesiąc. Powinien normalizować do~proporcji
    (np.~\% inicjacji danej osoby), aby nie mylić spadku aktywności
    z~brakiem inicjatywy.
    \index{normalizacja trendów}
\end{enumerate}
\end{featurebox}


% ============================================================
\section{Podsumowanie rozdziału}
\label{sec:quant-summary}

\begin{scorebox}[title={Ocena końcowa silnika ilościowego: \scoremed{6{,}5/10}}]

\textbf{Metodologia oceny:} Analiza kodu źródłowego wszystkich modułów
silnika ilościowego, weryfikacja matematyczna wzorów, ocena trafności
psychologicznej metryk, identyfikacja edge case'ów i~błędów logicznych.

\vspace{8pt}

\begin{tabularx}{\textwidth}{l c X}
  \toprule
  \textbf{Kategoria} & \textbf{Ocena} & \textbf{Komentarz} \\
  \midrule
  Architektura & \scorehi{8/10} & O(n) single-pass, modularna struktura \\
  Poprawność matematyczna & \scoremed{6/10} & 5~błędów krytycznych, 8~istotnych \\
  Trafność psychologiczna & \scorelo{5/10} & Behavioral proxies $\neq$ psychologia \\
  Kompletność metryk & \scoremed{6/10} & Brak sentymentu, tematów, konfliktów \\
  Jakość kodu & \scorehi{8/10} & Czytelny, dobrze typowany, modularny \\
  Obsługa edge case'ów & \scoremed{5/10} & Brak clampowania, złe domyślne \\
  \bottomrule
\end{tabularx}
\end{scorebox}

\begin{strengthbox}[title={Kluczowe mocne strony}]
\begin{itemize}
  \item \textbf{Architektura O(n) single-pass}~--- wydajna, czytelna,
    z~jasnym podziałem na~inicjalizację, pętlę główną i~post-processing.
  \item \textbf{60+ metryk}~--- imponujące pokrycie behawioralne
    bez żadnego udziału AI.
  \item \textbf{Platform-aware}~--- osobne session gaps dla Discord,
    fallbacki na~mentions/replies.
  \item \textbf{CPS (63~pytania)}~--- solidna baza do~screeningu wzorców
    komunikacyjnych.
  \item \textbf{Modularny kod TypeScript}~--- dobrze typowany,
    czytelny, z~komentarzami ``why''.
  \item \textbf{ReciprocityIndex}~--- eleganckie formuły $1 - 2|x - 0{,}5|$,
    łatwe do~interpretacji.
\end{itemize}
\end{strengthbox}

\begin{criticalbox}[title={Kluczowe problemy do~naprawienia}]
\begin{itemize}
  \item \bugid{QUANT-01}: Niepoprawny mianownik reactionRate w~grupach.
  \item \bugid{QUANT-02} + \bugid{QUANT-05}: Brak danych $\to$ ``perfekcja''
    zamiast wartości neutralnej.
  \item \bugid{QUANT-04}: delusionHolder semantycznie odwrócony.
  \item Metryki wiralne (Compatibility, Interest, Delusion) wymagają
    walidacji psychometrycznej i~uzupełnienia o~wymiary sentymentalne.
  \item 8~magic numbers bez uzasadnienia empirycznego (1200, 25, 500, 0{,}6 itp.).
\end{itemize}
\end{criticalbox}
