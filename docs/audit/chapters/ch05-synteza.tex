% ============================================================
% Rozdział 5 — Synteza Ekspercka
% ============================================================

\chapter{Synteza Ekspercka}
\label{ch:synteza}

\begin{center}
\Large\itshape\color{PodBlue}
,,Prawdziwy wgląd nie pochodzi z~jednego źródła --- powstaje na skrzyżowaniu perspektyw.''
\end{center}

\vspace{8pt}

Niniejszy rozdział nie powtarza wyników agentów, lecz identyfikuje \textbf{wzorce przekrojowe}, sprzeczności między raportami i~systemowe problemy, które ujawniają się dopiero przy syntezie trzech niezależnych analiz.

% ============================================================
\section{Metodologia syntezy}
\label{sec:synteza-metodo}

Synteza została przeprowadzona w~trzech krokach:
\begin{enumerate}
  \item \textbf{Ekstrakcja}: zebranie wszystkich znalezionych problemów, mocnych stron i~rekomendacji z~raportów trzech agentów.
  \item \textbf{Krzyżowa weryfikacja}: porównanie, czy ten sam problem został zauważony przez wielu agentów, oraz identyfikacja sprzeczności.
  \item \textbf{Kategoryzacja przekrojowa}: grupowanie wyników w~tematy systemowe, które wykraczają poza zakres pojedynczego agenta.
\end{enumerate}

% ============================================================
\section{Wzorce przekrojowe}
\label{sec:wzorce-przekrojowe}
\index{wzorce przekrojowe}

\subsection{Wzorzec 1: Brak walidacji na wielu poziomach}

Problem braku walidacji pojawia się \textbf{niezależnie we wszystkich trzech raportach}, ale na różnych warstwach:

\begin{table}[H]
\centering
\caption{Brak walidacji --- manifestacja per warstwa}
\label{tab:brak-walidacji}
\begin{tabularx}{\textwidth}{l l X}
\toprule
\textbf{Warstwa} & \textbf{Agent} & \textbf{Problem} \\
\midrule
Dane wejściowe & \agentthree & Rate limiting wyłączony, relationshipContext bez walidacji enum \\
Algorytmy & \agentone & Zero-data zwraca 100 zamiast neutralne 50 (brak guard clause) \\
Psychometria & \agenttwo & Health Score arbitralne wagi, brak odniesienia do walidowanych instrumentów \\
Parsery dat & \agentthree & DD/MM vs MM/DD heurystyka może zawodzić dla dat dwuznacznych \\
Scoring AI & \agenttwo & Subtext decoder: ad-hoc punktacja bez podstawy lingwistycznej \\
\bottomrule
\end{tabularx}
\end{table}

\begin{warningbox}[title=Wniosek]
System konsekwentnie \emph{zakłada poprawność} danych wejściowych i~parametrów zamiast je walidować. Dotyczy to zarówno warstwy technicznej (brak guard clause), jak i~merytorycznej (brak walidacji psychometrycznej).
\end{warningbox}

\subsection{Wzorzec 2: Napięcie rozrywka--nauka}
\index{rozrywka vs nauka}

Drugi wzorzec jest centralnym dylematem produktowym \podtekst. Z~jednej strony, moduły rozrywkowe (Roast, Court Trial, Dating Profile) stanowią o~unikalności produktu i~napędzają wiralność. Z~drugiej, łączenie psychologii klinicznej (attachment theory, Big Five, manipulation detection) z~komedią tworzy ryzyko:

\begin{itemize}
  \item \textbf{\agenttwo}: ,,Roasting weaponizes attachment styles --- mixing psychology with comedy risks trivializing serious patterns.''
  \item \textbf{\agenttwo}: ,,Labels like BAZOWANY vs TOTAL DELULU are funny but not psychological.''
  \item \textbf{\agentone}: ,,Treat viral scores as entertainment, not relationship diagnosis.''
\end{itemize}

\begin{moderatebox}[title=Implikacja]
Użytkownicy mogą internalizować wyniki (np.\ ,,nasz Health Score to 52, więc relacja jest chora'') bez zrozumienia, że te liczby \textbf{nie są zwalidowanymi instrumentami klinicznymi}. Disclaimery istnieją, ale są krótkie i~łatwe do przeoczenia.
\end{moderatebox}

\subsection{Wzorzec 3: Architektura dojrzała, infrastruktura niedojrzała}

\begin{table}[H]
\centering
\caption{Kontrast dojrzałości: architektura vs infrastruktura}
\begin{tabularx}{\textwidth}{L{5cm} C{2cm} L{5cm} C{2cm}}
\toprule
\textbf{Dojrzałe (8+/10)} & \textbf{Ocena} & \textbf{Niedojrzałe ($<$6/10)} & \textbf{Ocena} \\
\midrule
Silnik O(n) & \scorehi{8.0} & Walidacja psychometryczna & \scorelo{3.0} \\
Obsługa błędów & \scorehi{8.5} & Bezpieczeństwo AI & \scorelo{5.5} \\
Schematy JSON & \scorehi{9.0} & Rate limiting & \scorelo{2.0} \\
Kalibracja kontekstu & \scorehi{8.0} & Disclaimery & \scorelo{5.0} \\
Parsery platform & \scorehi{7.5} & Entertainment framing & \scorelo{4.0} \\
\bottomrule
\end{tabularx}
\end{table}

System jest \textbf{technicznie dojrzały} (TypeScript strict, czyste wzorce, wyrafinowane algorytmy), ale \textbf{produktowo i~naukowo niedojrzały} (brak walidacji, brak auth/payments, brak testów automatycznych).

\subsection{Wzorzec 4: Doskonała separacja klient--serwer}
\index{prywatność!separacja klient-serwer}

\begin{strengthbox}[title=Konsensus wszystkich agentów]
Wszystkie trzy agenty niezależnie potwierdziły, że architektura prywatności jest \textbf{wzorcowa}:
\begin{itemize}
  \item Surowe wiadomości przetwarzane wyłącznie po stronie klienta (przeglądarka).
  \item Do serwera trafia jedynie 200--500 spróbkowanych wiadomości per pass (< 1\% rozmowy).
  \item IndexedDB lokalna, localStorage z~prefiksem \texttt{podtekst-*}.
  \item Share URL anonimizuje wszystkie dane osobowe + kompresja LZ-string.
  \item Discord token użyty jednorazowo, nie przechowywany.
\end{itemize}
\end{strengthbox}

% ============================================================
\section{Sprzeczności między agentami}
\label{sec:sprzecznosci}
\index{sprzeczności}

\subsection{CPS: ,,Strong'' vs ,,6/10 frameworks''}

\agentone ocenił CPS Communication Pattern Screening jako \emph{,,najsilniejszy komponent psychologiczny''}, podczas gdy \agenttwo przyznał frameworkom psychologicznym ogółem \scoremed{6/10}.

\textbf{Rozstrzygnięcie}: Nie ma sprzeczności. \agentone oceniał CPS \emph{relatywnie} (na tle innych metryk wiralnych --- Compatibility 3/10, Interest 3/10, Delusion 2/10). \agenttwo oceniał \emph{absolutnie} (wobec standardów psychometrii klinicznej). CPS jest rzeczywiście najlepszym komponentem systemu, ale nawet on nie przeszedł walidacji psychometrycznej.

\subsection{Parsery ,,7--8.5/10'' vs ,,5 critical bugs''}

\agentthree ocenił parsery wysoko (Messenger 8.5, WhatsApp 7.5), a~jednocześnie \agentone znalazł 5~bugów krytycznych w~silniku ilościowym, który działa \emph{na danych z~parserów}.

\textbf{Rozstrzygnięcie}: Parsery i~silnik ilościowy to dwie różne warstwy. Parsery poprawnie normalizują surowe dane do \tstype{UnifiedMessage}. Bugi są w~warstwie \emph{post-processing} (viral-scores.ts, reciprocity.ts, wrapped-data.ts), nie w~parsowaniu. Nie ma sprzeczności.

% ============================================================
\section{Gotowość produkcyjna}
\label{sec:gotowosc}
\index{gotowość produkcyjna}

\begin{table}[H]
\centering
\caption{Traffic light --- gotowość per obszar}
\label{tab:traffic-light}
\begin{tabularx}{\textwidth}{L{4.5cm} C{2cm} X}
\toprule
\textbf{Obszar} & \textbf{Status} & \textbf{Warunek przejścia} \\
\midrule
\rowcolor{PodSuccess!8}
Parsery platform & \scorehi{GO} & Gotowe do produkcji \\
\rowcolor{PodSuccess!8}
Silnik ilościowy (core) & \scorehi{GO} & Po naprawie 5~bugów krytycznych \\
\rowcolor{PodSuccess!8}
Pipeline AI (4 pasy) & \scorehi{GO} & Po zmianie safety settings \\
\rowcolor{PodWarning!8}
Moduły rozrywkowe & \scoremed{WAIT} & Dodać disclaimery, review framing \\
\rowcolor{PodWarning!8}
Interfejs użytkownika & \scoremed{WAIT} & Decompose analysis page, a11y fixes \\
\rowcolor{PodDanger!8}
Bezpieczeństwo & \scorelo{STOP} & Rate limiting, safety settings, enum validation \\
\rowcolor{PodDanger!8}
Walidacja psychologiczna & \scorelo{STOP} & Health Score redesign, disclaimery \\
\rowcolor{PodDanger!8}
Auth \& Payments & \scorelo{STOP} & Supabase + Stripe (nie zaimplementowane) \\
\bottomrule
\end{tabularx}
\end{table}

% ============================================================
\section{Dług techniczny}
\label{sec:dlug}
\index{dług techniczny}

Na podstawie trzech raportów zidentyfikowano następujące kategorie długu technicznego:

\begin{enumerate}
  \item \textbf{Brak testów automatycznych}: Zero unit/integration tests. Vitest skonfigurowany (Faza 21), ale brak pokrycia. \emph{Ryzyko: regresje przy każdej zmianie.}
  \item \textbf{Monolityczna strona wyników}: 985~LOC w~jednym pliku (\filepath{analysis/[id]/page.tsx}). \emph{Ryzyko: trudność w~utrzymaniu, testowaniu i~code review.}
  \item \textbf{Hardcoded multipliers}: Minimum 6~magicznych liczb (1200, 25, 500, 200, 300, 3x) bez kalibracji. \emph{Ryzyko: arbitralność wyników.}
  \item \textbf{Brak CI/CD}: Brak automatycznego pipeline'u budowania, testowania i~deploymentu. \emph{Ryzyko: błędy trafiają na produkcję.}
  \item \textbf{Rate limiting wyłączony}: Funkcja istnieje, ale jest wyłączona w~kodzie. \emph{Ryzyko: abuse API, koszty Gemini.}
  \item \textbf{Preview model Gemini}: Użycie \texttt{gemini-3-flash-preview} (niestabilny preview) zamiast stabilnej wersji. \emph{Ryzyko: zmiany w~API mogą złamać pipeline.}
\end{enumerate}

% ============================================================
\section{Co warto dodać --- analiza przyszłościowa}
\label{sec:przyszlosc-kontent}
\index{przyszłe funkcje}

Na podstawie analizy brakujących metryk (\agentone), luk w~frameworkach psychologicznych (\agenttwo) i~możliwości UI (\agentthree), identyfikujemy następujące obszary wartego rozwoju:

\subsection{Nowe metryki analityczne}

\begin{featurebox}[title=Brakujące metryki (priorytet: wysoki)]
\begin{description}
  \item[Analiza sentymentu (LIWC-based)] Detekcja pozytywnego/negatywnego/neutralnego tonu per wiadomość. Pozwala na śledzenie trajektorii emocjonalnej rozmowy w~czasie. Bazować na LIWC (Linguistic Inquiry and Word Count) lub polskim odpowiedniku.
  \item[Detekcja konfliktów] Automatyczne rozpoznawanie kłótni, eskalacji i~rozwiązań. Sygnały: nagły wzrost długości wiadomości, zmiana tonu, cisza > 24h po intensywnej wymianie.
  \item[Wspólne tematy (topic modeling)] Wyekstrahowanie głównych tematów rozmowy i~śledzenie, które tematy inicjuje każda osoba. LDA lub embedding-based clustering.
  \item[Progresja bliskości/wrażliwości] Detekcja, czy rozmowa z~czasem staje się bardziej osobista (dłuższe wiadomości, większa wrażliwość, mniej defensywności).
\end{description}
\end{featurebox}

\subsection{Nowe funkcje rozrywkowe}

\begin{featurebox}[title=Proponowane nowe moduły]
\begin{description}
  \item[Couple Mode (tryb par)] Wspólna analiza dwóch osób z~porównaniem perspektyw. Każda osoba odpowiada na quiz i~widzi, jak postrzega relację vs jak dane ją opisują.
  \item[Team Features (funkcje zespołowe)] Dla grup 5+ osób: role (lider, mediator, prowokator), podgrupy, koalicje, izolowani członkowie. Rozszerzenie metryki sieci.
  \item[Shared Analyses] Możliwość udostępnienia pełnej analizy drugiej osobie z~rozmowy (za zgodą) zamiast tylko anonimowej karty.
  \item[Longitudinal Tracking] Powtórzenie analizy tej samej rozmowy po miesiącach/latach z~porównaniem trendów.
  \item[Voice Message Transcription] Transkrypcja wiadomości głosowych (Whisper API) i~włączenie do analizy tekstowej.
\end{description}
\end{featurebox}

\subsection{Walidacja naukowa}

\begin{infobox}[title=Rekomendowane badania walidacyjne]
\begin{enumerate}
  \item \textbf{Benchmarking Big Five}: Porównanie wyników z~kwestionariuszem NEO-PI-R na próbie 100+ osób, które dostarczą zarówno eksporty rozmów, jak i~wypełnione kwestionariusze.
  \item \textbf{Walidacja Health Score}: Porównanie z~Couples Satisfaction Index (CSI) lub Relationship Assessment Scale (RAS) --- czy wysoki Health Score koreluje z~wysoką satysfakcją?
  \item \textbf{A/B testing disclaimer'ów}: Test, czy bardziej wyraźne disclaimery (,,To NIE jest diagnoza kliniczna'') wpływają na interpretację wyników przez użytkowników.
  \item \textbf{Kalibracja progów CPS}: Porównanie wyników CPS z~diagnozami klinicznymi na próbie pacjentów terapii par.
\end{enumerate}
\end{infobox}

% ============================================================
\section{Podsumowanie syntezy}
\label{sec:synteza-podsumowanie}

\podtekst jest \textbf{technicznie imponującym produktem} z~dojrzałą architekturą kodu, innowacyjnym pipeline'em AI i~solidnym podejściem do prywatności. Główne ryzyka leżą nie w~kodzie, lecz w~\textbf{interpretacji wyników}: użytkownicy mogą traktować entertainment-grade metryki jako clinical-grade diagnozy.

Priorytetami powinny być: (1)~naprawa 9~bugów krytycznych, (2)~włączenie zabezpieczeń (rate limiting, safety settings), (3)~dodanie wyraźnych disclaimerów, (4)~walidacja psychometryczna kluczowych metryk. Dopiero po tych krokach system będzie gotowy do skalowania.
