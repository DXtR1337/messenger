\documentclass[a4paper,11pt]{article}

% ============================================================
% Packages
% ============================================================
\usepackage[utf8]{inputenc}
\usepackage[T1]{fontenc}
\usepackage[polish]{babel}
\usepackage{geometry}
\geometry{margin=2cm}
\usepackage{hyperref}
\hypersetup{
  colorlinks=true,
  linkcolor=blue!60!black,
  urlcolor=blue!60!black,
  pdftitle={PodTeksT — Kompletna Dokumentacja Funkcji},
  pdfauthor={Claude Opus 4.6},
}
\usepackage{booktabs}
\usepackage{longtable}
\usepackage{tabularx}
\usepackage{xcolor}
\usepackage{enumitem}
\usepackage{fancyhdr}
\usepackage{titlesec}
\usepackage{amsmath}

% ============================================================
% Custom colors
% ============================================================
\definecolor{ptblue}{HTML}{3B82F6}
\definecolor{ptpurple}{HTML}{A855F7}
\definecolor{ptgreen}{HTML}{10B981}
\definecolor{ptred}{HTML}{EF4444}
\definecolor{ptyellow}{HTML}{F59E0B}
\definecolor{ptbg}{HTML}{0A0A0A}
\definecolor{ptcard}{HTML}{111111}

% ============================================================
% Header/Footer
% ============================================================
\pagestyle{fancy}
\fancyhf{}
\fancyhead[L]{\footnotesize\textcolor{gray}{PodTeksT — Dokumentacja Funkcji}}
\fancyhead[R]{\footnotesize\textcolor{gray}{v2.0 | Faza 24}}
\fancyfoot[C]{\thepage}
\renewcommand{\headrulewidth}{0.4pt}

% ============================================================
% Title formatting
% ============================================================
\titleformat{\section}{\Large\bfseries\color{ptblue}}{\thesection}{1em}{}
\titleformat{\subsection}{\large\bfseries\color{ptpurple}}{\thesubsection}{1em}{}
\titleformat{\subsubsection}{\normalsize\bfseries}{\thesubsubsection}{1em}{}

% ============================================================
% Document
% ============================================================
\begin{document}

% ============================================================
% Title Page
% ============================================================
\begin{titlepage}
\centering
\vspace*{4cm}

{\Huge\bfseries\textcolor{ptblue}{Pod}\textcolor{ptpurple}{Teks}\textcolor{ptblue}{T}}

\vspace{0.8cm}
{\large\itshape odkryj to, co kryje się między wierszami}

\vspace{2cm}
{\LARGE\bfseries Kompletna Dokumentacja Funkcji}

\vspace{0.5cm}
{\large Wszystkie funkcje, metryki, zakresy i algorytmy}

\vspace{3cm}

\begin{tabular}{rl}
  \textbf{Wersja:} & 2.0 (Faza 24) \\
  \textbf{Data:} & \today \\
  \textbf{AI Development:} & Claude Opus 4.6 \\
  \textbf{AI Analysis:} & Gemini 3 Flash Preview \\
  \textbf{Stack:} & Next.js 16+, TypeScript, Tailwind v4 \\
\end{tabular}

\vfill
{\footnotesize Wygenerowano automatycznie na podstawie kodu źródłowego}
\end{titlepage}

% ============================================================
% Table of Contents
% ============================================================
\tableofcontents
\newpage

% ============================================================
% 1. WPROWADZENIE
% ============================================================
\section{Wprowadzenie}

\textbf{PodTeksT} to aplikacja webowa analizująca eksporty rozmów z~komunikatorów. Użytkownik wgrywa plik eksportu, a~aplikacja oblicza \textbf{80+ metryk ilościowych} (client-side, bez AI) oraz przeprowadza \textbf{wieloprzebiegową analizę psychologiczną AI} (server-side, Gemini API).

\subsection{Obsługiwane platformy}
\begin{itemize}[nosep]
  \item Facebook Messenger (JSON)
  \item WhatsApp (TXT)
  \item Instagram DM (JSON)
  \item Telegram (JSON)
  \item Discord (API — import przez bota)
\end{itemize}

\subsection{Architektura}
\begin{itemize}[nosep]
  \item \textbf{Parsing + analiza ilościowa:} Client-side (przeglądarka) — surowe wiadomości nigdy nie opuszczają urządzenia
  \item \textbf{Analiza jakościowa AI:} Server-side API routes — wysyłane 200--500 próbkowanych wiadomości per przebieg
  \item \textbf{Przechowywanie danych:} IndexedDB + localStorage (lokalne, bez serwera)
  \item \textbf{Streaming:} SSE (Server-Sent Events) z heartbeatem co 15s
\end{itemize}

% ============================================================
% 2. ANALIZA ILOŚCIOWA
% ============================================================
\section{Analiza ilościowa (80+ metryk)}

Wszystkie metryki w~tej sekcji są obliczane \textbf{wyłącznie po stronie klienta}, bez udziału AI. Obliczenia wykonywane w~jednym przebiegu $O(n)$ przez moduł \texttt{quantitative.ts}.

% ---- 2.1 Per-person ----
\subsection{Statystyki podstawowe (per osoba)}

\begin{longtable}{p{4.5cm}p{6cm}p{3cm}}
\toprule
\textbf{Metryka} & \textbf{Opis} & \textbf{Typ / Zakres} \\
\midrule
\endhead
totalMessages & Całkowita liczba wiadomości & \texttt{number} $\geq 0$ \\
totalWords & Całkowita liczba słów & \texttt{number} $\geq 0$ \\
avgWordsPerMessage & Średnia słów na wiadomość & \texttt{number} $\geq 0$ \\
longestMessage & Najdłuższa wiadomość (słowa) & \texttt{number} $\geq 0$ \\
shortestMessage & Najkrótsza wiadomość (słowa) & \texttt{number} $\geq 1$ \\
questionCount & Liczba pytań (znaki ?) & \texttt{number} $\geq 0$ \\
questionRatio & Odsetek pytań & \texttt{number} 0--1 \\
exclamationCount & Liczba wykrzykników & \texttt{number} $\geq 0$ \\
mediaCount & Multimedia (zdjęcia, video, GIF, stickery, audio, pliki) & \texttt{number} $\geq 0$ \\
reactionsSent & Reakcje wysłane & \texttt{number} $\geq 0$ \\
reactionsReceived & Reakcje otrzymane & \texttt{number} $\geq 0$ \\
reactionGiveRate & Rate reakcji dawanych & \texttt{number} 0--1 \\
reactionReceiveRate & Rate reakcji otrzymywanych & \texttt{number} 0--1 \\
deletedMessages & Usunięte wiadomości & \texttt{number} $\geq 0$ \\
editedMessages & Edytowane wiadomości & \texttt{number} $\geq 0$ \\
firstMessageDate & Data pierwszej wiadomości & \texttt{Date} \\
lastMessageDate & Data ostatniej wiadomości & \texttt{Date} \\
activeDays & Unikalne dni aktywności & \texttt{number} $\geq 1$ \\
topEmojis & Top 10 emoji per osoba & \texttt{[emoji, count][]} \\
topWords & Top 20 najczęstszych słów & \texttt{[word, count][]} \\
topPhrases & Top 10 fraz (2--4 słowa) & \texttt{[phrase, count][]} \\
\bottomrule
\end{longtable}

% ---- 2.2 Timing ----
\subsection{Timing i czasy odpowiedzi}

\begin{longtable}{p{5cm}p{5.5cm}p{3cm}}
\toprule
\textbf{Metryka} & \textbf{Opis} & \textbf{Typ / Zakres} \\
\midrule
\endhead
medianResponseTimeMs & Mediana czasu odpowiedzi (ms) & \texttt{number} $\geq 0$ \\
avgResponseTimeMs & Średni czas odpowiedzi (ms) & \texttt{number} $\geq 0$ \\
fastestResponseMs & Najszybsza odpowiedź (ms) & \texttt{number} $\geq 0$ \\
slowestResponseMs & Najwolniejsza odpowiedź (ms) & \texttt{number} $\geq 0$ \\
responseTimes[] & Tablica surowych czasów & \texttt{number[]} (ms) \\
conversationInitiations & Ile razy zainicjował rozmowę & \texttt{number} $\geq 0$ \\
conversationInitiationRatio & Odsetek inicjacji & \texttt{number} 0--1 \\
longestSilence & Najdłuższa cisza & \texttt{object} \\
\quad .durationMs & Czas trwania (ms) & \texttt{number} \\
\quad .startDate / endDate & Okres ciszy & \texttt{Date} \\
\quad .beforeSender / afterSender & Kto pisał przed/po & \texttt{string} \\
doubleTextRate & Odsetek podwójnych wiadomości & \texttt{number} 0--1 \\
doubleTextCount & Liczba double-textów & \texttt{number} $\geq 0$ \\
\bottomrule
\end{longtable}

\subsubsection{Dystrybucja czasów odpowiedzi (Response Time Distribution)}
Czasy odpowiedzi binowane w~8 kubełków:

\begin{center}
\begin{tabular}{cl}
\toprule
\textbf{Bin} & \textbf{Zakres} \\
\midrule
1 & $< 10$ sekund \\
2 & 10--30 sekund \\
3 & 30s -- 1 minuta \\
4 & 1--5 minut \\
5 & 5--15 minut \\
6 & 15--30 minut \\
7 & 30 min -- 1 godzina \\
8 & $> 1$ godziny \\
\bottomrule
\end{tabular}
\end{center}

% ---- 2.3 Engagement ----
\subsection{Zaangażowanie (Engagement)}

\begin{longtable}{p{5cm}p{5.5cm}p{3cm}}
\toprule
\textbf{Metryka} & \textbf{Opis} & \textbf{Typ / Zakres} \\
\midrule
\endhead
engagementScore & Ważony wynik zaangażowania & \texttt{number} 0--100 \\
consistencyScore & Stałość aktywności (std dev) & \texttt{number} $\geq 0$ \\
messageRatio & Stosunek wiadomości per osoba & \texttt{number} 0--1 \\
reactionRate & Rate reaktywności & \texttt{number} 0--1 \\
doubleTexts & Podwójne wiadomości & \texttt{Record<string, number>} \\
maxConsecutive & Max kolejnych wiadomości & \texttt{Record<string, number>} \\
avgConversationLength & Śr. długość rozmowy & \texttt{number} \\
totalSessions & Liczba sesji (przerwa > 6h) & \texttt{number} \\
lateNightRatio & Odsetek wiadomości 0:00--5:00 & \texttt{number} 0--1 \\
weekendRatio & Odsetek wiadomości weekend & \texttt{number} 0--1 \\
\bottomrule
\end{longtable}

% ---- 2.4 Heatmap ----
\subsection{Heatmapa aktywności}

\begin{itemize}[nosep]
  \item \textbf{Combined Heatmap:} Macierz $7 \times 24$ (dzień tygodnia $\times$ godzina), zsumowana po wszystkich uczestnikach
  \item \textbf{Per-person Heatmap:} Oddzielna macierz $7 \times 24$ dla każdego uczestnika
  \item \textbf{24-Hour Activity Chart:} Agregacja per godzina (0--23), stacked per osoba
\end{itemize}

% ---- 2.5 Trends ----
\subsection{Trendy miesięczne}

\begin{longtable}{p{5cm}p{5.5cm}p{3cm}}
\toprule
\textbf{Trend} & \textbf{Opis} & \textbf{Format} \\
\midrule
\endhead
responseTimeTrend & Mediana czasu odpowiedzi/miesiąc & \texttt{[month, perPerson]} \\
messageLengthTrend & Mediana długości wiadomości/miesiąc & \texttt{[month, perPerson]} \\
initiationTrend & Inicjacje rozmów/miesiąc & \texttt{[month, perPerson]} \\
monthlyVolume & Wolumen wiadomości/miesiąc & \texttt{[month, perPerson, total]} \\
sentimentTrend & Trend sentymentu/miesiąc & \texttt{[month, perPerson]} \\
volumeTrend & Kierunek zmiany wolumenu & \texttt{growing/declining/stable} \\
\bottomrule
\end{longtable}

\subsubsection{Year Milestones}
Obliczane z~\texttt{monthlyVolume}:
\begin{itemize}[nosep]
  \item \textbf{peakMonth:} Miesiąc z~największą liczbą wiadomości (nazwa PL + count)
  \item \textbf{worstMonth:} Miesiąc z~najmniejszą liczbą wiadomości
  \item \textbf{yoyTrend:} Zmiana rok-do-roku: $\frac{\text{suma nowsza połowa}}{\text{suma starsza połowa}} - 1$ (wartość procentowa)
\end{itemize}

% ---- 2.6 Sentiment ----
\subsection{Analiza sentymentu}

Obliczana per osoba na podstawie słownika pozytywnych/negatywnych słów:

\begin{longtable}{p{4.5cm}p{5.5cm}p{3.5cm}}
\toprule
\textbf{Metryka} & \textbf{Opis} & \textbf{Zakres} \\
\midrule
\endhead
positiveRatio & Odsetek pozytywnych wiadomości & 0--1 \\
negativeRatio & Odsetek negatywnych wiadomości & 0--1 \\
neutralRatio & Odsetek neutralnych & 0--1 \\
volatility & Zmienność emocjonalna (std dev) & $\geq 0$ \\
dominantTone & Dominujący ton & \texttt{string} \\
\bottomrule
\end{longtable}

% ---- 2.7 Conflicts ----
\subsection{Wykrywanie konfliktów}

Konflikty identyfikowane na podstawie nagromadzenia negatywnych słów kluczowych, krótkich gniewnych odpowiedzi i~eskalacji wzorców. Każdy konflikt zawiera:
\begin{itemize}[nosep]
  \item Timestamp konfliktu
  \item Uczestnicy
  \item Severity (niski/średni/wysoki)
  \item Słowa-wyzwalacze
\end{itemize}

% ---- 2.8 Intimacy ----
\subsection{Progresja intymności}

Miesięczny wynik intymności (0--100) obliczany z~kombinacji:
\begin{itemize}[nosep]
  \item Długość wiadomości (dłuższe = bardziej intymne)
  \item Aktywność nocna (late-night sharing)
  \item Częstotliwość i~czas odpowiedzi
  \item Wzajemność (reciprocity)
\end{itemize}
Fazy: \texttt{casual} $\rightarrow$ \texttt{deepening} $\rightarrow$ \texttt{intimate}

% ---- 2.9 Pursuit-Withdrawal ----
\subsection{Pursuit-Withdrawal Detection}

Wykrywanie cyklicznych wzorców pogoni i~wycofania:

\begin{itemize}[nosep]
  \item \textbf{Pursuit:} 3+ kolejnych wiadomości od tej samej osoby w~oknie $< 30$ minut
  \item \textbf{Withdrawal:} Następna odpowiedź po $> 2$ godzinach ciszy
  \item \textbf{Wymagane:} Minimum 2 cykle do identyfikacji wzorca
\end{itemize}

Wynik:
\begin{longtable}{p{4.5cm}p{5.5cm}p{3.5cm}}
\toprule
\textbf{Pole} & \textbf{Opis} & \textbf{Typ} \\
\midrule
\endhead
pursuer & Kto goni & \texttt{string} \\
withdrawer & Kto się wycofuje & \texttt{string} \\
cycleCount & Liczba wykrytych cykli & \texttt{number} $\geq 2$ \\
avgCycleDurationMs & Średni czas wycofania & \texttt{number} (ms) \\
escalationTrend & Trend eskalacji & \texttt{number} ($>0$ = pogarsza się) \\
\bottomrule
\end{longtable}

% ---- 2.10 Reciprocity ----
\subsection{Indeks wzajemności (Reciprocity Index)}

Kompozytowy wynik 0--1 z~4 składników:
\[
\text{RI} = 0.3 \cdot r_{\text{msg}} + 0.25 \cdot r_{\text{word}} + 0.25 \cdot r_{\text{init}} + 0.2 \cdot r_{\text{rt}}
\]
gdzie:
\begin{itemize}[nosep]
  \item $r_{\text{msg}}$ — stosunek wiadomości (bliżej 0.5 = lepiej)
  \item $r_{\text{word}}$ — stosunek słów
  \item $r_{\text{init}}$ — stosunek inicjacji rozmów
  \item $r_{\text{rt}}$ — stosunek czasów odpowiedzi
\end{itemize}

% ---- 2.11 Burst Detection ----
\subsection{Burst Detection}

Identyfikacja nagłych wzrostów aktywności:
\begin{itemize}[nosep]
  \item \textbf{Próg:} 10+ wiadomości w~oknie 5 minut
  \item \textbf{Wynik:} Lista burst events z~timestampem, liczbą wiadomości, uczestnikami
\end{itemize}

% ---- 2.12 Viral Scores ----
\subsection{Viral Scores}

\begin{longtable}{p{4cm}p{6cm}p{3.5cm}}
\toprule
\textbf{Score} & \textbf{Formuła} & \textbf{Zakres} \\
\midrule
\endhead
Compatibility Score & Ważona kombinacja: reciprocity, balance, response consistency, shared activity & 0--100 \\
Interest Score & Per osoba: initiation rate, response speed, message length ratio, question ratio & 0--100 \\
Ghost Risk & Per osoba: response time trend, silence frequency, message gap growth & 0--100 \\
Delusion Score & Per osoba: |self-perception - actual behavior| across 5+ metrics & 0--100 \\
\bottomrule
\end{longtable}

% ---- 2.13 Threat Meters ----
\subsection{Threat Meters (Mierniki zagrożeń)}

4 kompozytowe metryki zagrożeń (0--100):

\begin{longtable}{p{3.5cm}p{6.5cm}p{3.5cm}}
\toprule
\textbf{Miernik} & \textbf{Składniki} & \textbf{Severity Labels} \\
\midrule
\endhead
Ghost Risk & Z~\texttt{viralScores.ghostRisk} & Niski / Umiarkowany / Podwyższony / Krytyczny \\
Codependency & Initiation imbalance, double-text rate, response time asymmetry & j.w. \\
Manipulation & CPS control + manipulation + passive aggression patterns & j.w. \\
Trust & Reciprocity (odwrócony), response consistency, ghost frequency & j.w. \\
\bottomrule
\end{longtable}

% ---- 2.14 Damage Report ----
\subsection{Damage Report}

\begin{longtable}{p{4cm}p{6cm}p{3.5cm}}
\toprule
\textbf{Metryka} & \textbf{Formuła} & \textbf{Zakres} \\
\midrule
\endhead
Emotional Damage & $100 - \text{healthScore} + \text{sentiment asymmetry modifier}$ & 0--100\% \\
Communication Grade & Z~reciprocity: A($>80$), B($>60$), C($>40$), D($>20$), F($\leq 20$) & A--F \\
Repair Potential & Green flags count + response consistency + sentiment trend slope & 0--100\% \\
Therapy Needed & HS $< 40$ $\rightarrow$ TAK, $< 60$ $\rightarrow$ ZALECANE, else NIE & TAK/ZALECANE/NIE \\
\bottomrule
\end{longtable}

% ---- 2.15 Ranking Percentiles ----
\subsection{Ranking Percentiles (heurystyczne)}

Percentyle obliczane z~aproksymacji CDF rozkładu log-normalnego (Abramowitz \& Stegun):
\[
\Phi(z) \approx 1 - d \cdot t \cdot (a_1 + t(a_2 + t(a_3 + t(a_4 + t \cdot a_5))))
\]
gdzie $z = \frac{\ln(\text{value}) - \ln(\text{median})}{\sigma}$ i $t = \frac{1}{1 + 0.2316419|z|}$.

\begin{longtable}{p{4cm}p{3.5cm}p{2cm}p{4cm}}
\toprule
\textbf{Metryka} & \textbf{Mediana ref.} & \textbf{$\sigma$} & \textbf{Wynik} \\
\midrule
\endhead
Message Volume & 3000 wiad. & 1.2 & TOP X\% \\
Response Time & 480 000 ms (8 min) & 1.0 & TOP X\% (odwrócony) \\
Ghost Frequency & 12h & 0.8 & TOP X\% \\
Asymmetry & 20 & 0.9 & TOP X\% \\
\bottomrule
\end{longtable}

Kolorowanie: \textcolor{ptyellow}{TOP 10\% = złoty}, TOP 25\% = srebrny, reszta = szary.

% ---- 2.16 Badges ----
\subsection{Odznaki (Badges)}

15 typów odznak automatycznie przyznawanych:

\begin{longtable}{p{3.5cm}p{7cm}p{3cm}}
\toprule
\textbf{Badge} & \textbf{Warunek} & \textbf{Emoji} \\
\midrule
\endhead
Night Owl & $>$20\% wiadomości 0:00--5:00 & 🦉 \\
Chatterbox & $>$10000 wiadomości & 💬 \\
Double Texter & $>$15\% double-text rate & 📱📱 \\
Emoji King/Queen & $>$30\% wiadomości z~emoji & 👑 \\
Ghost Buster & Szybki medianowy czas odpowiedzi & 👻 \\
Essay Writer & Średnio $>$30 słów/wiadomość & 📝 \\
Question Master & $>$20\% pytań & ❓ \\
Speed Demon & Mediana odpowiedzi $< 60$s & ⚡ \\
Conversation Starter & $>$60\% inicjacji & 🎯 \\
Media Lover & $>$20\% wiadomości z~mediami & 📸 \\
Weekend Warrior & $>$40\% wiadomości w~weekend & 🎉 \\
Early Bird & $>$20\% wiadomości 5:00--8:00 & 🐦 \\
Storyteller & Regularnie długie wiadomości & 📖 \\
Reactor & Wysoki reaction rate & ❤️ \\
Marathon Chatter & Długie sesje rozmów & 🏃 \\
\bottomrule
\end{longtable}

% ---- 2.17 Catchphrases ----
\subsection{Catchphrases i Best Time to Text}

\begin{itemize}[nosep]
  \item \textbf{Catchphrases:} Top 5 unikalnych fraz per osoba, mierzonych częstotliwością i~unikalnością (fraza używana nieproporcjonalnie często przez jedną osobę)
  \item \textbf{Best Time to Text:} Optymalna godzina i~dzień tygodnia na podstawie najwyższego response rate i~najkrótszego response time
\end{itemize}

% ---- 2.18 Cognitive Functions ----
\subsection{Cognitive Functions Clash}

Deterministyczne mapowanie MBTI $\rightarrow$ 4 funkcje kognitywne (Jung):

Każdy z~16 typów MBTI mapowany na stos: \texttt{dominant}, \texttt{auxiliary}, \texttt{tertiary}, \texttt{inferior}. Przykład:
\begin{itemize}[nosep]
  \item ENFJ: Fe (dom) → Ni (aux) → Se (ter) → Ti (inf)
  \item INTP: Ti (dom) → Ne (aux) → Si (ter) → Fe (inf)
\end{itemize}

\textbf{Clash analysis:} Porównanie dominant/auxiliary obu osób — clash score 0--100 per para funkcji z~opisem konfliktu.

% ---- 2.19 Gottman ----
\subsection{Czterech Jeźdźców Gottmana}

Mapowanie z~CPS patterns + sygnałów ilościowych:

\begin{longtable}{p{3cm}p{7cm}p{3.5cm}}
\toprule
\textbf{Jeździec} & \textbf{Źródła} & \textbf{Severity} \\
\midrule
\endhead
Krytyka & CPS: kontrola + egocentryzm & none/mild/moderate/severe \\
Pogarda & CPS: manipulacja + asymetria odpowiedzi & j.w. \\
Defensywność & CPS: pasywna agresja + podejrzliwość & j.w. \\
Stonewalling & CPS: unikanie bliskości + dystans emocjonalny + ghost risk & j.w. \\
\bottomrule
\end{longtable}

Risk level: Niski (0 aktywnych) → Umiarkowany (1) → Podwyższony (2) → Wysoki (3) → Krytyczny (4).

% ============================================================
% 3. ANALIZA AI (GEMINI API)
% ============================================================
\section{Analiza AI (Gemini API)}

Wieloprzebiegowa analiza psychologiczna z~wykorzystaniem modelu \texttt{gemini-3-flash-preview}. Każdy przebieg otrzymuje 200--500 próbkowanych wiadomości. Wagi próbkowania: ostatnie 3 miesiące = 60\%.

\subsection{Pass 1 — Przegląd relacji}

\begin{longtable}{p{4.5cm}p{9cm}}
\toprule
\textbf{Pole} & \textbf{Opis} \\
\midrule
\endhead
tone & Ogólny ton rozmowy (formalny, swobodny, napięty...) \\
communicationStyle & Styl komunikacji (bezpośredni, pasywny, żartobliwy...) \\
relationshipType & Typ relacji (romantyczna, przyjacielska, rodzinna...) \\
dominantThemes & Główne tematy rozmów \\
languageComplexity & Złożoność języka per osoba \\
\bottomrule
\end{longtable}

\subsection{Pass 2 — Dynamika relacji}

\begin{longtable}{p{4.5cm}p{9cm}}
\toprule
\textbf{Pole} & \textbf{Opis} \\
\midrule
\endhead
powerBalance & Bilans władzy w~relacji \\
conflictPatterns & Wzorce konfliktów \\
intimacyLevel & Poziom intymności \\
emotionalLabor & Rozkład pracy emocjonalnej \\
redFlags[] & Czerwone flagi z~cytatami dowodowymi \\
greenFlags[] & Zielone flagi z~cytatami \\
turningPoints[] & Punkty zwrotne na timeline \\
\bottomrule
\end{longtable}

\subsection{Pass 3 — Profile indywidualne}

Osobny profil dla każdego uczestnika:

\begin{longtable}{p{4.5cm}p{9cm}}
\toprule
\textbf{Wymiar} & \textbf{Opis} \\
\midrule
\endhead
Big Five & 5 wymiarów: openness, conscientiousness, extraversion, agreeableness, neuroticism (każdy 0--100) \\
MBTI & 4-literowy typ + confidence (0--100) \\
Attachment Style & secure / anxious / avoidant / disorganized + opis \\
Love Languages & Ranking 5 języków miłości (scored) \\
Communication Style Meters & directness, formality, emotionality, humor (0--100) \\
Tone Radar & warm, cold, sarcastic, supportive, aggressive, passive (0--100) \\
\bottomrule
\end{longtable}

\subsection{Pass 4 — Synteza końcowa}

\begin{longtable}{p{4.5cm}p{9cm}}
\toprule
\textbf{Pole} & \textbf{Opis} \\
\midrule
\endhead
healthScore & Health Score 0--100 z~5 sub-scores \\
\quad communication & Sub-score: komunikacja & \\
\quad emotional & Sub-score: emocje & \\
\quad balance & Sub-score: balans & \\
\quad growth & Sub-score: rozwój & \\
\quad stability & Sub-score: stabilność & \\
redFlags[] & Czerwone flagi z~evidence & \\
greenFlags[] & Zielone flagi z~evidence & \\
turningPoints[] & Kluczowe momenty na timeline & \\
recommendations & Rekomendacje dla relacji & \\
predictions[] & Prognozy AI (patrz niżej) & \\
\bottomrule
\end{longtable}

\subsubsection{AI Predictions}
Każda predykcja zawiera:
\begin{itemize}[nosep]
  \item \texttt{prediction:} Treść prognozy (tekst)
  \item \texttt{confidence:} Pewność 0--100\%
  \item \texttt{timeframe:} Horyzont czasowy (np. ``Q1 2025'')
  \item \texttt{basis:} Uzasadnienie na podstawie trendów
\end{itemize}
Disclaimer: \textit{Predykcje AI oparte na trendach — nie stanowią pewności.}

% ============================================================
% 4. MODUŁY INTERAKTYWNE
% ============================================================
\section{Moduły interaktywne (AI-powered)}

\subsection{Enhanced Roast}
Psychologiczny roast z~pełnym kontekstem Pass 1--4. Endpoint: \texttt{/api/analyze} (roast pass w~głównym SSE).

\subsection{Stand-Up Comedy Roast}
7-aktowa rozmowa w~stylu stand-up comedy. Generuje PDF do pobrania.
\begin{itemize}[nosep]
  \item Endpoint: \texttt{POST /api/analyze/standup}
  \item Streaming: SSE z~7 aktami
  \item Export: jsPDF client-side
\end{itemize}

\subsection{CPS Screener (Communication Pattern Screening)}
63 pytania analizujące 10 wzorców komunikacyjnych:

\begin{longtable}{p{0.5cm}p{5cm}p{2.5cm}p{2.5cm}}
\toprule
\textbf{\#} & \textbf{Wzorzec} & \textbf{Pytań} & \textbf{Próg} \\
\midrule
\endhead
1 & Unikanie bliskości & 6 & $\geq 50\%$ \\
2 & Nadmierna kontrola & 7 & $\geq 50\%$ \\
3 & Kontrola i~perfekcjonizm & 6 & $\geq 50\%$ \\
4 & Podejrzliwość i~nieufność & 7 & $\geq 50\%$ \\
5 & Egocentryzm komunikacyjny & 6 & $\geq 50\%$ \\
6 & Intensywność emocjonalna & 7 & $\geq 50\%$ \\
7 & Dramatyzacja i~histrionizm & 6 & $\geq 50\%$ \\
8 & Manipulacja i~brak empatii & 6 & $\geq 50\%$ \\
9 & Emocjonalny dystans & 6 & $\geq 50\%$ \\
10 & Pasywna agresja & 6 & $\geq 50\%$ \\
\bottomrule
\end{longtable}

\subsection{Subtext Decoder (Translator Podtekstów)}
AI dekoduje ukryte znaczenia w~wiadomościach.
\begin{itemize}[nosep]
  \item Kontekst: 30+ wiadomości na wymianę
  \item Kategorie: deflection, hidden\_anger, seeking\_validation, power\_move, genuine, testing, guilt\_trip, passive\_aggressive, love\_signal, insecurity, distancing, humor\_shield
  \item Output: oryginał → dekodowana wiadomość + confidence + kategoria
  \item Deception Score per osoba (0--100\%)
\end{itemize}

\subsection{Court Trial (Twój Chat w~Sądzie)}
AI prowadzi rozprawę sądową z~pełnym procesem:
\begin{itemize}[nosep]
  \item Akt oskarżenia z~konkretnymi zarzutami
  \item Mowa oskarżycielska z~cytatami z~rozmowy
  \item Mowa obrończa z~okolicznościami łagodzącymi
  \item Wyrok: \texttt{winny} / \texttt{niewinny} / \texttt{warunkowo}
  \item Kara (kreatywna, dopasowana do ``przestępstwa'')
  \item Share card: Mugshot Card z~sygnaturą sprawy
\end{itemize}

\subsection{Dating Profile Generator}
Szczery profil randkowy na podstawie zachowań tekstowych:
\begin{itemize}[nosep]
  \item Bio na podstawie stylu komunikacji
  \item Age Vibe (mentalny wiek)
  \item Statystyki kluczowe (szybkość reakcji, inicjatywa, cierpliwość)
  \item Prompts \& Answers (w~stylu Hinge)
  \item Red Flags + Green Flags
  \item Overall Rating
  \item Share card: Gazeta Podtekstowa (styl ogłoszenia matrymonialnego)
\end{itemize}

\subsection{Reply Simulator}
Symulacja odpowiedzi w~głosie partnera:
\begin{itemize}[nosep]
  \item Użytkownik wpisuje wiadomość
  \item AI odpowiada w~stylu wybranego uczestnika
  \item Bazuje na: stylu komunikacji, tonie, częstych frazach, response patterns
\end{itemize}

\subsection{Delusion Quiz}
Test samoświadomości — porównanie percepcji z~rzeczywistością:
\begin{itemize}[nosep]
  \item Seria pytań o~własne zachowania w~rozmowie
  \item Odpowiedzi porównywane z~rzeczywistymi danymi ilościowymi
  \item \textbf{Delusion Index:} 0--100 (im wyżej, tym większa rozbieżność)
  \item Kategorie: Low Delusion ($< 30$), Moderate ($30$--$60$), High ($> 60$)
\end{itemize}

\subsection{Couple's Quiz}
Quiz wiedzy o~partnerze (osobna trasa \texttt{/couple}):
\begin{itemize}[nosep]
  \item Pytania generowane z~danych ilościowych
  \item Odpowiedzi weryfikowane z~rzeczywistymi statystykami
  \item Wynik kompatybilności
\end{itemize}

% ============================================================
% 5. SHARE CARDS
% ============================================================
\section{Share Cards (24+ typów)}

Karty do udostępniania w~mediach społecznościowych, eksportowane jako PNG.

\begin{longtable}{p{4.5cm}p{6cm}p{3cm}}
\toprule
\textbf{Karta} & \textbf{Zawartość} & \textbf{Styl} \\
\midrule
\endhead
StatsCard & Kluczowe statystyki rozmowy & Ciemny, numeryczny \\
PersonalityCard & Profil osobowości & Gradient \\
PersonalityPassportCard & Paszport osobowości & Paszportowy \\
ReceiptCard & Paragon z~relacji & Paragonowy \\
VersusCard / V2 & Porównanie 1v1 & Splitscreen \\
CompatibilityCardV2 & Wynik kompatybilności & Gradientowy \\
RedFlagCard & Czerwone flagi & Czerwono-czarny \\
GhostForecastCard & Prognoza ghostingu & Mglisty \\
HealthScoreCard & Health Score & Gauge \\
MBTICard & Typ MBTI & Kolorowy \\
BadgesCard & Odznaki & Grid \\
FlagsCard & Red + Green Flags & Dwukolorowy \\
LabelCard & Etykieta relacji & Minimalistyczny \\
ScoresCard & Viral Scores & Neonowy \\
SubtextCard & Dekodowane wiadomości & Matrix/wojskowy \\
MugshotCard & Mugshot z~wyroku & Policyjny \\
DatingProfileCard & Profil randkowy & Gazetowy \\
DelusionCard & Wynik Delusion Quiz & Brutalny \\
SimulatorCard & Symulacja odpowiedzi & Chatowy \\
CoupleQuizCard & Wynik Couple's Quiz & Romantyczny \\
CPSCard & Rentgen komunikacji & Medyczny/RTG \\
\bottomrule
\end{longtable}

Każda karta: 360px szerokości, minimalna wysokość 640px, eksport do PNG via \texttt{html-to-image}.

% ============================================================
% 6. STORY MODE (WRAPPED)
% ============================================================
\section{Story Mode (Wrapped)}

12-scenowa prezentacja w~stylu Spotify Wrapped:
\begin{enumerate}[nosep]
  \item Intro z~animacją
  \item Kluczowe liczby (total messages, days, words)
  \item Osobowości uczestników
  \item MBTI porównanie
  \item Versus stats
  \item Red flags parade
  \item Roast highlight
  \item Court verdict
  \item Interactive recap
  \item Stand-Up fragment
  \item Heatmap visual
  \item Wrapped summary
\end{enumerate}

Animacje: Framer Motion, auto-play z~możliwością nawigacji.

% ============================================================
% 7. SERVER VIEW
% ============================================================
\section{Server View (5+ uczestników)}

Aktywacja: \texttt{isGroup \&\& participants.length > 4}

Dodatkowe komponenty w~trybie grupowym:
\begin{itemize}[nosep]
  \item \textbf{PersonNavigator:} Nawigacja między profilami (20-kolorowa paleta)
  \item \textbf{PersonProfile:} Indywidualny profil per osoba
  \item \textbf{ServerLeaderboard:} Ranking uczestników
  \item \textbf{PairwiseComparison:} Porównanie 1v1 dowolnej pary
  \item \textbf{ServerOverview:} Przegląd statystyk grupowych
\end{itemize}

\textit{W~trybie grupowym ukryte:} Viral Scores, Ghost Forecast, Delusion Quiz.

% ============================================================
% 8. DISCORD BOT
% ============================================================
\section{Discord Bot (11 komend)}

Bot Discord z~11 komendami slash. Infra: HTTP interactions (Ed25519 verification), in-memory channel cache (1h TTL, 50 entries LRU).

\begin{longtable}{p{3cm}p{7cm}p{3.5cm}}
\toprule
\textbf{Komenda} & \textbf{Opis} & \textbf{Parametry} \\
\midrule
\endhead
/stats & Statystyki kanału & limit? \\
/versus & Porównanie 2 osób & user1, user2 \\
/whosimps & Kto simpi na kogo & limit? \\
/ghostcheck & Kto ghostuje & limit? \\
/besttime & Najlepszy czas na pisanie & -- \\
/catchphrase & Catchphrases uczestników & -- \\
/emoji & Top emoji per osoba & -- \\
/nightowl & Nocne marki & -- \\
/ranking & Ranking aktywności & -- \\
/roast & AI roast kanału & target? \\
/personality & AI profil osobowości & target \\
\bottomrule
\end{longtable}

% ============================================================
% 9. EKSPORT
% ============================================================
\section{Eksport i~udostępnianie}

\begin{itemize}[nosep]
  \item \textbf{PDF (standard):} Pełny raport analizy z~wykresami via jsPDF
  \item \textbf{PDF (Stand-Up):} 7-aktowa komedia w~PDF
  \item \textbf{Share Cards:} 24+ typów kart PNG
  \item \textbf{Web Share API:} Natywne udostępnianie na mobile
  \item \textbf{Public Share Links:} Zakodowane linki z~podsumowaniem (bez surowych wiadomości)
\end{itemize}

% ============================================================
% 10. API
% ============================================================
\section{API Endpoints}

\begin{longtable}{p{5cm}p{2cm}p{4cm}p{2.5cm}}
\toprule
\textbf{Endpoint} & \textbf{Method} & \textbf{Opis} & \textbf{Rate Limit} \\
\midrule
\endhead
/api/analyze & POST & 4-pass AI analysis (SSE) & 5/10min \\
/api/analyze/enhanced-roast & POST & Enhanced roast (SSE) & 5/10min \\
/api/analyze/standup & POST & Stand-Up 7 acts (SSE) & 5/10min \\
/api/analyze/cps & POST & CPS 63 questions (SSE) & 5/10min \\
/api/analyze/subtext & POST & Subtext Decoder (SSE) & 5/10min \\
/api/analyze/court & POST & Court Trial (SSE) & 5/10min \\
/api/analyze/dating-profile & POST & Dating Profile (SSE) & 5/10min \\
/api/analyze/simulate & POST & Reply Simulator (SSE) & 5/10min \\
/api/analyze/image & POST & Image generation & 10/10min \\
/api/discord/fetch-messages & POST & Discord message fetch (SSE) & 3/10min \\
/api/health & GET & Health check & brak \\
\bottomrule
\end{longtable}

% ============================================================
% 11. PRYWATNOŚĆ
% ============================================================
\section{Prywatność i~bezpieczeństwo}

\begin{enumerate}[nosep]
  \item Surowe wiadomości przetwarzane client-side — nigdy nie przesyłane na serwer
  \item Tylko 200--500 próbkowanych wiadomości wysyłanych do Gemini API per pass
  \item Brak treści rozmów w~logach serwera
  \item Wszystkie dane w~przeglądarce (IndexedDB / localStorage)
  \item GDPR-friendly — użytkownik może usunąć dane lokalnie
  \item Rate limiting na wszystkich endpointach API
  \item CSP headers (Content Security Policy)
\end{enumerate}

% ============================================================
% FOOTER
% ============================================================
\vfill
\begin{center}
\rule{0.5\textwidth}{0.4pt}

\vspace{0.5cm}
{\small\textcolor{gray}{PodTeksT — odkryj to, co kryje się między wierszami}}

{\footnotesize\textcolor{gray}{Wygenerowano: \today{} | Claude Opus 4.6 | Faza 24}}
\end{center}

\end{document}
